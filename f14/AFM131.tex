\documentclass[english, 12pt]{article}
\usepackage{yingconfig}
% ========================Variables======================================
\newcommand{\coursecode}{AFM 131}
\newcommand{\coursename}{Introduction to Business in North America}
\newcommand{\thisprof}{Prof. Sproule}
\newcommand{\curterm}{Fall 2014} 
\toggletrue{completed}

\begin{document}

\notesheader

\section{Globalization}
Canada's market only consists of 34.8 million people, but there are over 7 billion potential customers in the global market.
\begin{defn}
\textbf{Exporting} is selling goods and services to other countries. \textbf{Importing} is buying products from other countries. Competition is intense in exports.
\end{defn}
\subsection{Why trade?}
Global trade allows countries to increase production efficiency, and product diversity. This is beneficial to both parties.
\begin{defn}
\textbf{Comparative advantage theory}: A country should export products that it produces more effectively and efficiently than others, and import products that cannot be produced efficiently and effectively.
\end{defn}
\begin{defn}
\textbf{Absolute advantage}: A country has an advantage when it can have a monopoly on producing a specific product or is able to produce it more efficiently than everyone else.
\end{defn}
\subsubsection*{Conditions for imports into Canada}
\begin{enumerate}
\item Prohibited (eg: hate literature)
\item Allowed only under authority of import permit (clothing, steel, wheat, chicken, eggs, firearm, etc)
\item Federal imposed condition? (labeling law, emission control standards, sanitary checks)
\item Privately-certified standard. (eg electronics)
\item Provincial rules (liquor, wine)
\end{enumerate}
\subsection{Measuring global trade}
\begin{defn}
\textbf{Balance of trade} is the nation's ratio of exports to imports.
\end{defn}
\begin{defn}
\textbf{Balance of payments} is exports subtract money leaving a country (imports, tourism, military, foreign investment)
\end{defn}
Canada is extremely dependent on the US as a trading partner, but is moving towards other priority markets such as south-east Asia, Australia, Brazil, China, India, Europe, etc.
\subsection{Reaching Global Markets}
\subsubsection*{Licensing}
A firm may license the right to manufacture its product or use its trademark to a foreign company (licensee) for a fee (royalty).The licensee attaches their own labelling to the product and is restricted to where they c an sell the product. The licensor pays little money on marketing but carries risk of losing money because product is dominating in foreign region, or product secrets exposure.
\subsubsection*{Exporting}
Exporting is selling directly into a foreign market, sometimes utilizing export-trading companies that can assist in negotiations and establishing trade relationships.
\subsubsection*{Franchising}
Sale of a right to use a business name and sell it in a given territory. Franchiser has a large portion of the control (compared to licensing) of the franchisee's business operations.

\subsubsection*{Contract manufacturing}
Foreign company makes a domestic company's products (outsourcing), attach the company's brand, and directly sell the product in other countries.

\subsubsection*{International Joint Ventures \& Strategic Alliance}
Partnership between domestic and foreign company. Companies pool technology, marketing, management, and knowledge to share risk of selling in a market. Companies form strategic alliances to build their own competitive market advantages without sharing costs, risks, and profits

\subsubsection*{Foreign Investment}
Directly buying permanent property and businesses in foreign nations. A foreign subsidiary is a company owned by a parent company. A company can only become multinational if it manufactures and markets products in many different countries, and has multinational stock ownership and management.

\subsection{Forces affecting trade in Global Markets}
\begin{defn}
Sociocultural differences such as values, beliefs, attitudes, religion, language, rules, and institutions may impact the ability to conduct business and manage employees.
\end{defn}
\begin{defn}
Economic and financial forces include consumer wealth and exchange rates. \textbf{Devaluation} is when a country intentionally lowers the value of its currency to increase export potential. \textbf{Countertrading} is when multiple countries are involved in a trade.
\end{defn}
\begin{defn}
Legal and regulatory forces are unique to different countries and must be understood to be successful.
\end{defn}

\begin{defn}
Physical and environment forces may restrain a company's decisions.
\end{defn}

\subsection{Trade Protectionism}
Use of government regulations to limit the import of goods and services.
\begin{defn}
\textbf{Dumping} involves selling products abroad at lower prices to get rid of surplus.
\end{defn}
\begin{defn}
Tariffs are taxes imposed on imports.
\end{defn}
\begin{defn}
\textbf{Protective tariffs} raise the retail price so domestic products will be more competitively produced. Saves jobs for domestic workers. \textbf{Revenue tariffs} raise money for the government
\end{defn}
\begin{defn}
\textbf{Import quota} is a limit of the number of products in a certain category that can be imported to preserve jobs and prevent dumping.
\end{defn}
\begin{defn}
\textbf{Embargo} bans an import or export of a certain product or bans a country.
\end{defn}
The International Monetary Fund supports countries with a balance of trade problem. Less-developed countries with developing infrastructure are aided by the World Bank.
\begin{defn}
\textbf{Producers' cartels} are organizations of commodity producing countries to stabilize or increase prices on specific commodities.
\end{defn}
\subsubsection*{North American Free Trade Agreement}
Formed by Canada, USA, and Mexico to eliminate trade barriers, promote conditions of fair competition, increase investment opportunities, provide effective protection and enforcement of intellectual property, establish a framework for trade cooperation, and improve working conditions.
\subsubsection*{European Union}
Began as an alliance of 6 trading partners. Today, consists of 27 member nations and serves the same purpose as the NAFTA.
\subsubsection*{EU Debt Crisis}
A handful of smaller European countries overspent and over-borrowed. Several countries have already received bailout funds from the IMF. However, even with the funding these small economies could not service their debt as interest rates increased. This threatens the euro as these countries may discard it and return to their own currencies. It is still an ongoing crisis that is yet to be resolved.
\subsection{Future}
Global trade opportunities become ore interesting, yet challenging, each day. The internet has transformed the landscape of business. E-commerce enables companies to bypass historical distribution channels, and reach a large market in a few mouse clicks.
\subsection*{Further Reading}
\subsubsection*{Six Potential Stumbling Blocks to Canada-EU Trade Deal}
\begin{itemize}
\item \textbf{Automobile} - Origin of content. Most made in Canada vehicles are assembled in Canada, but components are from US or Mexico. Sometimes only 25\% Canadian content. Europe wants more Canadian content if they want to bypass tariffs.
\item \textbf{Agriculture} - France and Ireland may lose market share to Canadian beef and pork
\item \textbf{Intellectual Property} - EU wants Canada to extend copyright protections by two years for brand name drugs.
\item \textbf{Financial Services} - EU wants their banks to operate more freely in Canada. Canada objects because will create competition with Canadian banks.
\item \textbf{Provincial Procurement} - Provinces and territories will not be able to prevent foreign bidding on local projects if deal goes through.
\item \textbf{Canada Investment Act} - EU wants to bypass the act, but that allows for foreign takeover.
\end{itemize}
\subsubsection*{Borderline Insanity}
Ironic that Canada did not hesitate to bail out auto companies like GM and Chrysler (to save Canadian auto industry), but only had a single customs clearance to import a shipment of cars from offshore companies. If a Canadian business had a large amount of US customers, a pre-clearance could be obtained and smooth out the clearance procedure.

\section{Government}
\subsection{Government Involvement in the Economy}
\begin{defn}
Canada has a \textbf{mixed economy}: resource allocation is done by both the market and the government.
\end{defn}
\begin{defn}
Canada also has a north to south trading configuration because most people live near the US border.
The US economy developed much faster than Canada's with a much larger population and economy. 
Many of their products were not available to the Canadian provinces because either they weren't made in Canada or Canada has a weak east/west distribution system.
This led to the federal government's development of the \textbf{National Policy}: setting high protective tariffs to protect Canadian manufacturing.
\end{defn}
\begin{defn}
An \textbf{industrial policy} is a comprehensive, coordinated government plan to guide and revitalize the economy.
\end{defn}
\subsection{Ways Government affects Businesses}
\subsubsection{Crown corporations}
Crown corporations were set up for a few reasons. Firstly, they provide services that were not being provided by businesses (Air Canada). Secondly, they bailed out a major industry in trouble (Canadian National Railway). finally, they provided special services that were not otherwise available (Bank of Canada)
\begin{defn}
\textbf{Privatization} is the process of governments selling Crown corporations.
\end{defn}
\begin{defn}
\textbf{Deregulation} is the government's withdrawal of regulations that hinder competition.
\end{defn}
\subsubsection{Laws and regulations}
Laws derived from the Constitution, precedents established by judges, provincial and federal statures, and federal and provincial administrative agencies.

Federal government's primary responsibility is economic performance and business operations. They are also responsible for overseeing various industries.
\begin{exmp}
Trade regulations, incorporation of federal companies, taxation, banking and monetary system, hospital insurance and medicare, public debt and property, national defense, unemployed, immigration, criminal law, fisheries.
\end{exmp}
Responsibilities overlap with provincial and territorial government on health sector.  Federal government funds, but provincial government implements and facilitates. (More info on pg 105)

The Competition Bureau ensures that Canadian businesses and consumers prosper in a competitive and innovative market. It is illegal to fix prices or restrict output. (Identical prices are not evidence of price fixing)
\begin{defn}
\textbf{Marketing boards} are organizations that control the supply or pricing of certain agricultural products. It gives stability to important areas of the economy that are volatile (weather, disease, unstable prices, uncoordinated farmers)
\end{defn}
To elevate Canadian farmers, the Canadian government can grant loans to countries to pay for wheat they buy from Canada. Canadian Wheat Board markets wheat and barley while other Boards control production through quotas.

Territories are governed federally while provinces have their own government. Provincial responsibilities include, regulation of provincial trade, natural resources, incorporation of provincial companies, taxation, licensing, administration of justice, health and social services, municipal affairs, property and labour law, education. 
\begin{defn}[PPP]
 \textbf{Public-private partnership} module has led to efficiency, cost-effectiveness (construction and maintenance), risk transfer to private partner.
 \end{defn}
 \begin{defn}
 \textbf{Agreement on Internal Trade} is an intergovernmental trade agreement to reduce barriers to the movement of persons, goods, services, and investment in Canada.
\end{defn}
Municipal government responsibilities are defined by the province. They provide services such as wager supply, sewage and garbage disposal, roads, sidewalks, street lighting and building codes. They also aid in consumer protection: serving food and zoning laws (noise, odours, and signs)

\subsubsection{Taxation and financial policies}
Taxes allow governments to redistribute wealth, discharge their legal obligations, pay for public services, pay debt, fund government operations, and encourage or discourage taxpayers and businesses.
\begin{defn}
\textbf{Fiscal policy} is the government's effort to stabilize the economy by increasing or decreasing taxes or government spending. Low taxes stimulate the economy by providing more profits for businesses, and higher spending. High taxes discourage small businesses, and slow down the economy.
\end{defn}
\begin{defn}
A \textbf{deficit} occurs when the government spends more money than the amount it gathers in taxes for a time period.
\end{defn}
\begin{defn}
\textbf{Monetary policy} is the management of the one supply and interest rates controlled by the Bank of Canada. Similar to fiscal policies, low interest encourages borrowing and spending, while high interests discourage borrowing and reduce spending.
\end{defn}
\begin{defn}
\textbf{Subprime mortgages} are loans targeteed at people who do not qualify for regular mortgages do to low or inexistent credit rating. In 2008 the US Congress had to conduct a \$700 billion bank bailout to stabilize the financial sector. After this incident, banks became more cautious.
\end{defn}
\subsubsection{Government expenditures}
Governments making many payments directly to individuals (grants) which increase their purchasing power (pensions, allowances, OSAP). They also spend ongoing money on various services such as education, health, roads, ports and airports to support businesses and individuals. Also intervene in ad hoc (special/ unplanned) issues.

\begin{exmp}
Governments provide a variety of direct assistance programs to businesses with a specific purpose in mind (save jobs). 
\end{exmp}
\begin{defn}
\textbf{Transfer payments} are direct payments from governments to other governments or individuals. (eg: employment insurance).
\end{defn}
\begin{defn}
\textbf{Equalization} is the federal government's transfer program to reduce fiscal disparities among provinces.
\end{defn}
\subsubsection{Purchasing policies}
Used to favour Canadian companies over foreign companies when government is purchasing goods and services. This occurs even if it might be more expensive, as it provides employment and potential reveals the advanced technology. Foreign companies are insisted to manufacture a minimum portion of their advanced technology in Canada.
\subsubsection{Services}
\textbf{Industry Canada} helps small businesses get smarted and their National Research Council supports research that helps Canadian industry remain competitie and innovative.\\\\
\textbf{Foreign Affairs and International Trade Canada} helps businesses with exporting and foreign investments (information, marketing expertise, finanial aid, insurance and guarantees, publications and suggested contract forms.) (See Fig 4.5 on pg 121)
\subsection*{Further Readings}
\subsubsection*{Health Canada}
Health Canada's role is to work towards ensuring food safety from products processed in Canada or imported. They must followed a certain set of standards in terms of regulation of facilities providing food, level of required inspections, labeling at both production and retail level, and testing. Rules that they must enforce are monitoring production facilities, auditing the production process, and impose penalties for not meeting their standards.
\subsubsection*{Maple Leaf Enhanced Safety Protocols}
Maple Leaf Foods protocols include food safety pledge, cleaning practices, sampling and testing, following their procedures, and having a Chief Food Safety Officer.
\subsubsection*{Issues with food inspection}
Should food producers be charged for cost of government inspection, and should it be the government's responsibility to inspect foods?
\subsubsection*{Keynes's Model}
Keynes uses a top down approach. Government spending can control the economy, and we can't just wait for the market to dictate itself. This model was implemented historically, and as a result, governments have massive debts.
\subsubsection*{Hayek's Model}
Hayek believes in bottom up approach. The consumers are ultimately the larger portion of the economy, and there is no need for government to intervene. The economy is able to sort itself out.
\section{Teamwork}
\subsection{I in Team}
An effective team member has the following criterias: 
\begin{itemize}
\item Commitment -  reliability and trustworthiness
\item Offer help and ask for help
\item Motivation - engagement with the team
\item Awareness of the value of the individual opinions and ensuring that all team members are heard.
\end{itemize}
An effective team member works toward the agreed goals, is enthusiastic, is committed, takes pride in their work, shows interest in other opinions, accepts feedback, is able to stay focused, openly communicates, is sensitive, is able to resolve conflicts, and most importantly, is open to new approaches.
\begin{note}
Bad experiences occur in teamwork when there is
\begin{itemize}
\item \textbf{Individualism} - People only thinking for themselves
\item \textbf{Control Freaks} - Overly aggressive leader, perfectionist.
\item \textbf{Unresolved Conflicts} - Differing opinions and no consensus.
\item \textbf{Freeloader} - That one useless guy in your team that you have to carry.
\end{itemize}
\end{note}
\subsection{Individual Differences}
Individual differences always exist whether it's personalities, cultural backgrounds, skill set, work styles, etc. There are some basic skills that come in handy when working in teams: Listening, assessment, feedback, negotiation, conflict resolution, coaching, and project management skills.
\subsection{Feedback}
Feedback is helpful because:
\begin{itemize}
\item Affirmative to the receiver
\item Motivation increased
\item Promotes learning
\item Improves relationship
\end{itemize}
\begin{note}
\textbf{Good feedback} involves discussing what you saw, heard, and felt. \textbf{Poor feedback} comes in when you talk what you think.
\end{note}
Steps to effectively utilize feedback:
\begin{itemize}
\item \textbf{Control} - Receiver has control of the feedback.
\item \textbf{Paraphrase} - Focus on understanding the feedback
\item \textbf{Correct misinformation or misperception} - If something is misunderstood, explain it.
\item \textbf{Express action} - Express what will be done about the problem.
\end{itemize}
\subsection{Team Meeting}
\textbf{Before the meeting}, ensure that all discussed tasks are completed, agenda is created and read, time and location is set, members will be present or their assigned task will be present.\\\\
\textbf{During the meeting}, arrive on time, review agenda, ensure everyone has participation opportunity, park unplanned items, and plan the next meeting.
\subsection*{Further Readings}
\subsubsection*{Ten is a Crowd}
Individuals are more effective than teams because they are more effective when creativity is necessary. In addition, tensions arise in teams, and they must be managed (thorough discussion) rather than resolved. Differing opinion sparks new ideas that should not be immediately discarded.
\subsubsection*{Teamwork and Collaboration - John Chambers}
John made huge changes to the management structure in CISCO. Both \textbf{command and control}, and \textbf{collaboration and teamwork} have their place: former is necessary for implementation, and latter is necessary for creation. The biggest challenge is for the CEO to let go of his authority and allow teams to figure things out without his dictation.
\subsubsection*{Tuckman's Model of Team Development}
\begin{itemize}
\item \textbf{Forming} - Rules established, team members meet each other.
\item \textbf{Storming} - Brainstorming ideas, usually some degree of resistance due to differences.
\item \textbf{Norming} - Things start to normalize as people start accepting the team and begin working.
\item \textbf{Performing} - Everyone becomes useful and finishes their task. Conflicts must be solved quickly for the team to progress smoothly.
\item \textbf{Adjourning} - Tasks are wrapped up and reflected on. Members usually say goodbye and peace to work on new projects.
\end{itemize}
\subsubsection*{Leading Teams - Hackman}
There are a few conditions to a successful team
\begin{itemize}
\item \textbf{Support} - Tasks, boundaries, authority, and stability must be set and understood by all team members before beginning any projects.
\item \textbf{Compelling Direction} - Goal must be understood by all members. Creates motivation.
\item \textbf{Enabling Structures} - The design, norms of conduct, and team composition of the team must be clearly identified.
\item \textbf{Supporting Context} - Organizational systems must be identified and understood. (Reward, Information, Educational)
\item \textbf{Expert Coaching} - Assessment of amount of effort, appropriateness of effort, and level of knowledge and skills.
\end{itemize}
\section{Accounting}
\subsection{Role of Accounting}
It is impossible to manage a business without being able to read, understanding or analyze accounting reports and financial statements.
\begin{defn}
\textbf{Accounting} is the recording, classifying, summarizing, and interpreting of financial events to provide different parties with the information needed to make decisions.
\end{defn}
\subsubsection*{Financial Events in Mike's Bike}
\begin{itemize}
\item Value of bike sales
\item Cost of bikes sold
\item Investment on advertising and public relations
\item Income tax
\item Net Income or Loss
\item Budget on Branding
\end{itemize}
\subsection{Accounting Disciplines}
Many accounting frauds have occurred in history, and the government now enforces stricter rules and regulations regarding accounting in the \textbf{Starbanes-Oxley Act} and \textbf{Bill 198}. There are three professional accounting designations: CA, CMA, and CPA.
\begin{defn}
\textbf{Managerial accounting} is internal and compares financial performance to budgets, other firms, past performances as a benchmark activity for planning and control purposes, as well as minimizing taxes.
\end{defn}
\begin{defn}
\textbf{Financial accounting} prepares information and analysis for external users such as creditors, employee unions, customers, suppliers, shareholders, government, etc. They are usually provided in the form of an annual report.
\end{defn}
\begin{defn}
\textbf{Compliance} involves reviewing and evaluating both internal and external records to ensure proper accounting and financial reporting procedures are followed.
\end{defn}
\begin{defn}
\textbf{Tax accounting} Prepares tax returns and develops tax strategies. Tax policies and regulations change, making it a challenging occupation.
\end{defn}
\begin{defn}
\textbf{Government and NPO accounting} checks whether the organization is fulfilling its obligations, and making proper use of the funds they are provided.
\end{defn}
\begin{defn}
\textbf{Private accountants} works for a single firm, government agency, or NPO.
\end{defn}
\begin{defn}
\textbf{Public accountants} are contracted by the business, and provides his or her services on a fee basis.
\end{defn}
\subsection{Accounting Cycle}
\begin{defn}
The \textbf{accounting cycle} is a six step procedure for preparing and analyzing financial statements.
\end{defn}
\begin{defn}
\textbf{Bookkeeping} is the recording of transactions.
\end{defn}
Computers and accounting software have greatly simplified the preparation of accounting records and financial reports.
\subsection{Understanding Financial Statements}
\begin{defn}
A \textbf{financial statement} is a summary of all the transactions that have occurred over a particular period. Three key financial statements: Balance sheet, Income statement, Cash flow statement.
\end{defn}
\subsubsection*{Mike's Bike Financial Statements}
\begin{itemize}
\item Mike's Bikes Market Research Reports: Market Information and Distribution Summary
\item Mike's Bikes Industry Reports: Industry Benchmark Report and Market Summary
\item Mike's Bikes Firm Results Reports: Income Statement, Balance Sheet, Cash Flow Statement and Cost of Goods Manufactured and Gross Margin Report
\item Mike's Bikes Product Marketing Reports: Products - Sales, Margin and Production Report
\item Mike's Bikes Operations Management Report: Factory Report 
\end{itemize}
\subsection{Applying Accounting Knowledge}
All financial statements must follow GAAP, but a new standard (IFRS) is growing in popularity. In Canada, LIFO is not permitted for inventory valuation.
\subsection{Ratio Analysis}
\begin{note}
Financial ratio standards may vary between different industries.
\end{note}
\begin{defn}
\textbf{Ratio analysis} is the assessment of a firm's financial position through calculation of the relationship between numbers reported compared to other ratios such as past results, competitors and industry benchmarks.
\end{defn}
\begin{rto}
\textbf{Liquidity} is how fast an asset can be  converted into cash. Current ratio measures short term ability to pay off debt. Acid test ratio (also known as quick ratio) consists of cash, receivables, and short-term investments, and gives a more accurate representation for firms with huge merchandise inventory. 
\end{rto}
\begin{rto}
\textbf{Leverage (Debt) Ratios} measures a firm's reliance on borrowed funds. Debt to Equity ratio is most common. Tax Payable is usually not included in D/E Ratio.
\end{rto}
\begin{rto}
\textbf{Profitability (Performance) Ratios} measure a firm's ability to use resources to earn profit. Consists of EPS, return on sales (net income/ net sales), and return on equity (net income/ average equity).
\end{rto}
\begin{rto}
\textbf{Activity Ratios} measure a firm's ability to turnover inventory. Inventory turnover ratio  is COGS/avg inventory.
\end{rto}

\section{Marketing}
\subsection{Introduction to Marketing}
\begin{defn}
\textbf{Marketing} is a set of business practices designed to plan for and present an organization's products or services in ways that build effective customer relationships. Another definition would be activities that buyers and sellers perform to facilitate mutually satisfying exchanges.
\end{defn}
\begin{defn}
A \textbf{market} is a group of people with unsatisfied needs or wants who have the willingness and capability to buy. It is created because of a demand for goods and services.
\end{defn}
\subsection*{Evolution of Marketing}
\begin{itemize}
\item \textbf{Production} - demands exceeded supply, and focus was on producing as much as possible.
\item \textbf{Sales} - supply caught up to demand, focused on persuading customers to buy.
\item \textbf{Marketing Concept} - Both demand and supply increased, focus on responsiveness to customers. Companies because customer, service, and profit oriented.
\item \textbf{Market Orientation} - collect information about customers needs and competitors capabilities and sharing the information throughout the organization. Aimed to strength relationships with customers.
\item \textbf{Social Media Marketing} - Building communities and networks to encourage participation and engagement with the company.
\end{itemize}
Non-profit organizations use marketing to support their goals. Some examples include donating blood, raising money, attracting members, raising awareness.
\subsection{Marketing Mix}
\begin{defn}
The \textbf{marketing mix} is a set of ingredients that go into a marketing program. Often referred to as the four Ps of marketing: product, price, place, promotion.
\end{defn}
\begin{exmp}

\textbf{The Marketing Process:}\\
Conduct research
$\rightarrow$ Identify a target market
$\rightarrow$ Design a product to meet the need based on research
$\rightarrow$ Concept testing
$\rightarrow$ Determine a brand name, design a package, and set a price
$\rightarrow$ Select distribution system
$\rightarrow$ Promotional Program
$\rightarrow$ Build relationships with customers
\end{exmp}

\begin{defn}
A \textbf{product} is any physical good, service, or idea that satisfies a need or want.
\end{defn}
\begin{defn}
\textbf{Test marketing} is the process of testing products among potential users
\end{defn}
\begin{defn}
A \textbf{brand name} is a word, letter, or mix that differentiates one seller's goods or services from those of competitors.
\end{defn}
\begin{defn}
\textbf{Price} is the money or other consideration exchanged for the ownership or use of a good or service.
\end{defn}
\begin{defn}
\textbf{Promotion} consists of all the techniques sellers use to inform people and motivate them to buy their product. It includes advertising, personal selling, public relations, direct marketing, and sales promotion.
\end{defn}
\subsection{Information in Marketing}
\begin{defn}
\textbf{Market research} is the analysis of markets to determine opportunities and challenges, and find the information needed to make good decisions.
\end{defn}
\subsubsection*{Market Research Process}
\begin{enumerate}
\item Defining the question and determining the situation
\item Collecting research data
\item Analyzing research data
\item Selecting a solution and implement it. Market research is a continuous process.
\end{enumerate}
\begin{defn}
A \textbf{focus group} is a small group of people who need under the direction of a discussion leader to communicate their opinions about an organization. 
\end{defn}
\begin{defn}
\textbf{Environmental scanning} is identifying the factors that can affect marketing success.
\end{defn}
\begin{itemize}
\item \textbf{Global} - use of Internet to reach customers leads to distribution consequences.
\item \textbf{Technological} - Build customer databases and produce customized goods.
\item \textbf{Social} -  social trends, specific demographic and ethnic groups, and population growth.
\item \textbf{Competitive} - Speed, service, price, selection
\item \textbf{Economic} - State of the global economy
\item \textbf{Legal} - Laws and regulations change, and may impact businesses.
\end{itemize}
\begin{defn}
\textbf{Consumer market} consists of individuals or households who obtain goods and services for personal use.
\end{defn}
\begin{defn}
\textbf{Business-to-business market} consists of organizations that use goods and services to produce other goods and services, or to sell, rent, supply goods to others.
\end{defn}
\subsection{Consumer Market}
\begin{defn}
\textbf{Market segmentation} is the process of dividing the total market into different groups whose members share a common characteristic. 
\end{defn}
\begin{defn}
\textbf{Target marketing} is marketing directed towards certain groups of people that can be served profitably.
\end{defn}
\subsubsection*{Types of Segmentation}
\begin{itemize}
\item \textbf{Geographic} - dividing the market by geographic area.
\item \textbf{Demographic} - dividing the market by age, income, and education.
\item \textbf{Psychographic} - dividing the market using their values, attitudes, and interests.
\item \textbf{Behavioural} - dividing the market based on behaviour with or toward a product.
\end{itemize}
\begin{note}
A combination of the segmentations should be used to create a consumer profile.
\end{note}
\begin{defn}
\textbf{Niche marketing} is finding a small but profitable market segment and designing or finding products for them.
\end{defn}
\begin{defn}
\textbf{One-to-one marketing} is developing a unique mix of goods and services for each individual customer.
\end{defn}
\begin{defn}
\textbf{Mass marketing} means developing products and promotion to pleasure a large audience.
\end{defn}
\begin{defn}
\textbf{Relationship marketing} consist of marketing strategies with the goal of keeping customers by offering them products that exactly meet their requirements.
\end{defn}
\subsubsection*{Consumer Decision Making Process}
\begin{enumerate}
\item Problem recognition
\item Search for information
\item Alternative evaluation
\item Purchase decision
\item Post-purchase evaluation
\end{enumerate}
\begin{defn}
\textbf{Cognitive dissonance} may occur when consumers make a major purchase has doubts about whether they got the best product at the best price.
\end{defn}
Various factors affect a consumer's decision making process:
\begin{itemize}
\item \textbf{Marketing Mix} - Product, price, place, promotion
\item \textbf{Psychological} - perceptions, attitudes, learning, motivation
\item \textbf{Situational} - type of purchase, social surroundings, physical surroundings, previous experiences.
\item \textbf{Sociocultural Influences} - Reference groups, family, class, culture (values), subculture (age, geography, ethnicity)
\end{itemize}
\begin{defn}
\textbf{Time poverty} occurs when the need for convenient products occurs due to lack of time.
\end{defn}
\begin{defn}
\textbf{Value-consciousness} is when people have a need to obtain the best quality, features and performance of a product for a given price.
\end{defn}
\subsection{Business-to-Business Market}
There are a very limited amount of customers in the B2B market. Furthermore, the size of business customers is huge and B2B markets are usually geographically concentrated. Business buyers are more rational and less emotional than consumers. B2B sales tend to be direct (not always) as they like to skip intermediaries. Lastly, there is much more emphasis on personal selling because there are fewer customers and that means they demand more personal service. The majority of B2B marketers engage in social media marketing as a large percentage of B2B buyers use the Internet for their purchase.
\subsection*{Further Readings}
\subsubsection*{The Rise of the Amateur Professional}
Consumers are the ultimate innovators behind products because they are the end users of the product. They are more knowledgeable about the product, and any innovations that improve the product will ultimately benefit them.
\subsubsection*{What is Customer Relationship Management}
Customer relationship management is a strategy for understanding your customers in order to optimize a company's interactions with them. Customer relation management systems can help suggest the best ways to interact with certain types of customers. Some of the information that may be tracked include
\begin{itemize}
\item What do they buy?
\item Order size and composition
\item Payment Terms and History
\item Preferred Sales Promotion
\end{itemize}
\subsubsection*{Interview with Peter Carr}
In the B2B market, some ways to approach retailers that aren't carrying your product include showing third party testimonials and consumer exposure, providing samples. When the product is first launched, extra support to the stores should be given. Examples of support include demo, discount programs, designing a good website, and providing sales support.
\section{Marketing II}
\subsection{Product Development}
\begin{defn}
When a product has \textbf{value} it means that the product is of good quality at a fair price. Benefits should exceed cost.
\end{defn}
\begin{defn}
\textbf{Total product offer} contains everything that consumers evaluate to see whether a product is worth it.
\end{defn}
\begin{exmp}
Potential Components of a Total Product Offer includes price, brand, convenience, package, store surroundings, service, internet, past experiences, guarantee, delivery, advertising, and reputation.
\end{exmp}
\begin{defn}
A \textbf{product line} is a group of products that are similar or intended for a similar market. \textbf{Product Mix} is the combination of product lines offered by a company.
\end{defn}
\begin{defn}
\textbf{Product differentiation} consists of the unique value enhancers that company use to make their product different or stand out compared to competitors' products. Examples include price, advertising, and packaging.
\end{defn}
\begin{exmp}
\textbf{Product packaging} must attract attention, protect the goods inside, deter theft, easy to open and use, provide information about contents, explain benefits, provide warnings and warranties, and give some indication of price, value, and uses.
\end{exmp}
\begin{exmp}
\textbf{Universal Product Codes} is a barcode that gives retailer information about the product's price, size,  color, and their attributes and helps control inventory.
\end{exmp}
\begin{exmp}
A \textbf{radio frequency identification chip} can track the location of a product at all times and carry more information than barcodes.
\end{exmp}
\begin{exmp}
\textbf{Bundling} is a strategy that combines multiple goods and services for a single price.
\end{exmp}

\subsection{Branding}
\begin{defn}
A \textbf{brand} is a name, symbol, or design, that identifies a seller.
\end{defn}
\begin{defn}
A \textbf{trademark} is a brand that has legal protection for its brand name and design.
\end{defn}
\begin{defn}
\textbf{Brand equity} is the value of a brand name, and is a measure of earning power.
\end{defn}
\begin{defn}
\textbf{Brand loyalty} is the degree to which customers are committed to further purchase of a brand.
\end{defn}
\begin{defn}
\textbf{Brand awareness} is how quickly a brand name comes to mind.
\end{defn}
\begin{defn}
A \textbf{brand manager} is responsible for a brand or product line and controls the marketing for that product line or brand.
\end{defn}
\subsubsection*{Product Life Cycle}
\begin{defn}
The \textbf{product life cycle} is a theoretical model of what happens to sales and profits for a product class over time. The stages are introduction, growth, maturity, and decline.
\end{defn}
\subsection{Pricing}
Pricing objectives include
\begin{itemize}
\item Achieve a target return on investment.
\item Build traffic - supermarkets sometimes under-price products, known as \textbf{loss leaders}, to attract customers to browse other contents of store.
\item Increase market share
\item Creating an image
\item Social Objectives
\end{itemize}
Major approaches to pricing:
\begin{itemize}
\item \textbf{Cost-based pricing} - Add up component prices, and add in profit margin.
\item \textbf{Demand-based pricing} - Meets profit margin and satisfies customers. Known as \textbf{target costing}.
\item \textbf{Competition-based pricing} - Price based on competitors. Could be higher, lower, or same depending on strategy.
\end{itemize}
\begin{defn}
\textbf{Price leadership} occurs when a dominant firm sets the pricing practices that all competitors in an industry follow.
\end{defn}
\begin{defn}
\textbf{Break-even analysis} is the process to determine profitability. The break-even point is the total fixed cost divided by the price of one unit less variable costs. $BEP = \f{FC}{P-VC}$.
\end{defn}
\begin{defn}
\textbf{Fixed Costs} are expenses that remain the same regardless of sales.
\end{defn}
\begin{defn}
\textbf{Variable costs} change according to the level of production. Includes production cost and labour cost.
\end{defn}
\begin{defn}
\textbf{Skimming price strategy} sets high prices to quickly recover costs.
\end{defn}
\begin{defn}
\textbf{Penetration price strategy} sets low prices to gain a larger market share. The problem with this is that prices cannot be raised later (to maintain customer base).
\end{defn}
\begin{defn}[EDLP]
\textbf{Everyday low pricing} is used by Home-Depot and Walmart, and involves setting prices lower than competitors and minimizing special sales.
\end{defn}
\begin{defn}
\textbf{High-low pricing} is used more often where regular prices are higher, but during special sales they become lower. It encourages customers to wait for sales.
\end{defn}
\begin{defn}
\textbf{Psychological pricing} makes products appear less expensive than it is. For example, $\$299.99$ instead of $\$300.00$.
\end{defn}
\subsection{Channels of Distribution}
\begin{defn}
\textbf{Market intermediaries} are organizations who assist in moving goods and services. A \textbf{channel of distribution} consists of a bunch of intermediaries. Intermediaries are responsible for transporting, storing, selling, advertising, and relationship building, and can do it more effectively than most manufacturers. (Comparative advantage).
\end{defn}
\begin{itemize}
\item \textbf{Agents/Brokers} - Brings buyer and seller together and negotiates exchange (real estate)
\item \textbf{Wholesalers} - Sells in the B2B market.
\item \textbf{ Retailer} - Sells to end consumers.
\end{itemize}
\subsubsection*{Retail Distribution Strategy}
\begin{itemize}
\item \textbf{Intensive} - spam retail outlets. (candy, magazines)
\item \textbf{Selective} - only a preferred group to ensure quality sales and service. (furniture, clothing)
\item \textbf{Exclusive} - only one retail outlet in a given geographic area. Even more customer service and brand awareness can be provided. (auto, specialty goods)
\end{itemize}
\subsubsection*{Non-store retailing}
\begin{itemize}
\item \textbf{Electronic retailing} - selling over internet. Issue with delivery, complaints and returns.
\item \textbf{Telemarketing} - selling by phone and providing a catalogue.
\item \textbf{Vending machines} - Dispense convenience goods. Benefit of location.
\item \textbf{Kiosks and Carts} - Lower overhead costs, built in a marketing atmosphere.
\item \textbf{Direct selling}  - selling directly to customers (home or workplace).
\end{itemize}
There are five major types of supply chains:
\begin{center}
\begin{tabular}{|c|c|c|c|c|c|c|}
\hline
Mode & Cost & Speed & On-Time & Handling & Frequency & Reach\\
\hline
Railroad & Medium & Slow & Medium & High & Low & High\\
Trucks & High & Fast & High & Medium & High & Most\\
Pipeline & Low & Medium & Highest & Lowest & Highest & Lowest\\
Ships & Lowest & Slowest & Lowest & Highest & Lowest & Low\\
Airplane & Highest & Fastest & Low & Low & Medium & Medium\\
\hline
\end{tabular}
\end{center}
\subsection{Promotion Mix}
\begin{defn}
\textbf{Promotion mix} is the combination of promotional tools organizations use.
\end{defn}
\begin{defn}
\textbf{Advertising} is paid and non-personal communication through various media.
\end{defn}
\begin{defn}
\textbf{Personal selling} is the face-to-face presentation and promotion of goods and services.
\end{defn}
\begin{defn}
\textbf{Public Relations} is the management function that evaluates public attitudes and requests toward a company. The aim is to execute a program to earn public acceptance.
\end{defn}
\begin{defn}
\textbf{Publicity} is information about an individual, product, or company that is distributed to the public through the media and is not controlled by the seller. It makes stories more believable and further reach, but negative publicity is detrimental as stories may be altered from the truth.
\end{defn}
\begin{defn}
\textbf{Sales promotion} are designed to supplement personal selling, advertising, and PR efforts by providing short-term activities to enhance a company or product's image. (Free sampling, coupons, sponsorships, trade shows, etc).
\end{defn}
\begin{defn}
\textbf{Direct marketing} is direct communication to consumers to hopefully generate a response for purchase.
\end{defn}
\begin{exmp}
\textbf{Word of mouth promotion} involves people telling others about products they purchased. \textbf{Buzz marketing} is the popularity of a product created by word of mouth.
\end{exmp}
\subsection{Managing the Promotion Mix}
\begin{defn}
\textbf{Push strategy} uses promotional tools to get wholesalers and retailers to carry your product so consumers can see it and hopefully buy it.
\end{defn}
\begin{defn}
\textbf{Pull strategies} use promotional tools to get consumers to go to stores and ask for products.
\end{defn}
\subsection*{Further Readings}
\subsubsection*{What we Learn From Spaghetti Sauce}
Although consumers have an incentive to innovate, they don't actually know what they want until they see it. Instead of asking them what they want, we need to show them what they want. For instance no one stated they want extra chunky spaghetti sauce in surveys, yet sampling results show that people have a preference for it.
\subsubsection*{Grub Guru}
Frito Lay Canada launched a program to ask Canadians to come up with a cachy name and commercial for a new chip flavour. The prize was only 25k and 1\% of sales , but the company was able to raise sales by 22\%. The success of Frito Lay can be replicated by carefully considering the elements of the contest, advertising, and aligning the elements with a targeted demographics' media habits.
\section{Operations}
\subsection{State of Canada}
Canada is facing issues in its ability to remain a competitive industrial country:
\begin{itemize}
\item Lagging Productivity
\item Inadequate Education/Retraining of Workforce
\item Foreign subsidiaries (profit not invested back in Canada)
\item Lack of Research and Development
\end{itemize}
\begin{defn}[R\&D]
\textbf{Research and Development} is work directed toward the innovation, introduction, and improvement of products and processes. Objective is to improve product quality, increase production capacity, and extend production range.
\end{defn}
\begin{note}
Service is a huge sector in our economy. To regain competitive edge, Canadian manufacturers focus on customers, quality, continuous improvement, the internet, and different production techniques. 
\end{note}
\subsection{Operations Management}
\begin{defn}
\textbf{Production management} describes the activities that managers do to help their firms create goods.
\end{defn}
\begin{defn}
\textbf{Operations management} converts or transforms resources into goods and services.
\end{defn}
\begin{note}
In the service sector, operations management is about creating a good user experience. Sometimes productivity is tied to improvements in quality which aren't measured such as speed of delivery and customer satisfaction. Productivity is also increased through the use of technology and automation.
\end{note}
\begin{defn}
\textbf{Facility location} is the process of selecting a geographic location for a company's operations. It can make products more accessible, and is influenced by availability of resources, access to transportation, proximity to suppliers, crime rates, quality of life, quality of workforce, and cost of living.
\end{defn}
\begin{defn}
\textbf{Facility layout} is the arrangement of resources to produce goods and provide services.
\end{defn}
\begin{itemize}
\item \textbf{Assembly-line:} works only do a few tasks
\item \textbf{Modular:} Teams of workers combine to produce more complex units
\item \textbf{Fixed position:} workers work around the product. (Big construction projects like a house)
\item \textbf{Process:} Similar equipment grouped together
\end{itemize}
\begin{defn}[SQC]
\textbf{Statistical quality control} evaluates quality at each stage of production.
\end{defn}
\begin{defn}[SPC]
\textbf{Statistical process control} takes samples at each production stage and plots it on a graph. It is then analyzed for variances.
\end{defn}
\begin{defn}
\textbf{Materials required planning} is an operations management system that uses sales forecasts to ensure inventory is sufficient.
\end{defn}
\begin{defn}
\textbf{Enterprise resource planning} enables firms to manage all operations (finance, planning, HR) and combines them into a single system.
\end{defn}
\begin{defn}
\textbf{Purchasing} is the functional area that searches for quality materials, best suppliers, and negotiates for a good price. Firms rely on fewer suppliers today because there is a lot of information shared between them.
\end{defn}
\begin{defn}[JIT]
\textbf{Just-in-Time inventory control} keeps inventory and minimum, and minimizes storage costs, obsolescence, theft, and damage costs. Problems include delayed shipment, and accurate production schedules.
\end{defn}
\subsection{Quality Control}
\begin{defn}
\textbf{Quality} is consistently producing what the customer wants and reducing production errors.
\end{defn}
\begin{defn}
\textbf{Six Sigma quality} is a quality measure that allows $3.4$ defects per million opportunities.
\end{defn}
\subsubsection*{Costs Associated with Quality}
\begin{itemize}
\item \textbf{Prevention} - prevent defects from occurring related to staff training
\item \textbf{Inspection} - Ensure quality inputs are added to production process
\item \textbf{Internal Failure} - Costs incurred with defective items before shipping to customer. (replacement)
\item \textbf{External Failure} - Costs incurred with defective items once sold to customer. (warranty, lose future sales)
\end{itemize}
\begin{defn}
The \textbf{International Organization for Standardization} is a worldwide federation of national standards.
\end{defn}
\begin{defn}
\textbf{ISO 9000} is the common name given to quality management and assurance standards.
\end{defn}
\begin{defn}
\textbf{ISO 14000} is a collection of the best practices for managing environmental impact.
\end{defn}
\begin{defn}
\textbf{ISO 26000} is a collection of standards on social responsibility.
\end{defn}
\begin{defn}
\textbf{Logistics} involves minimized costs to get products and services to the right place at the right time.
\end{defn}
\begin{defn}
A \textbf{supply chain} is a chain of firms that perform activities to create and deliver goods and services to the next personnel on the chain.
\end{defn}
\begin{defn}
\textbf{Supply chain management} is integrating logistics across firms in a supply chain.
\end{defn}
\subsection{Production Processes}
\begin{defn}
\textbf{Form utility} is the value added to the creation of finished goods and services.
\end{defn}
\begin{defn}
\textbf{Process manufacturing} is the part of production that physically or chemically changes materials.
\end{defn}
\begin{defn}
The \textbf{assembly process} combines different parts of a product together to form one product. (Automobile)
\end{defn}
\begin{defn}
\textbf{Continuous process} has a long production run that turns out to be finished goods over time. (Chemical Plant)
\end{defn}
\begin{defn}
\textbf{Intermittent process} production run is short, and machines are changed frequently to make different products.
\end{defn}
\subsection{Improvements}
\begin{defn}
\textbf{Flexible manufacturing} is designing machines to do multiple tasks.
\end{defn}
\begin{defn}
\textbf{Lean manufacturing} is producing goods using less of everything compared to mass production (labour, defects, space, investment, time).
\end{defn}
\begin{defn}
\textbf{Mass customization} is tailoring products to meet the needs of individuals.
\end{defn}
\begin{defn}
\textbf{Computer-aided design} and \textbf{computer-aided manufacturing} interact to form \textbf{computer-integrated manufacturing}, and it doubles productivity and allows for easy customizations.
\end{defn}
\subsection{Control Procedures}
\begin{defn}[PERT]
\textbf{Program evaluation and review technique} is a method to analyze the tasks and time involved in completing a given project.
\end{defn}
\begin{defn}
The \textbf{critical path} in PERT is the sequence of tasks that take the longest time to complete.
\end{defn}
\begin{defn}
A \textbf{Gantt chart} is a bar graph showing production managers their projects and each one's completion level.
\end{defn}
\subsection*{Further Readings}
\subsubsection*{Beating China on Cost}
Gildan Activewear Inc was able to compete against Chinese manufacturers by having shifting manufacturing to a cheap (Honduras is much closer than China) location with better technology than China, and taking advantage of trade agreements to eliminate tariffs and ship duty-free.
\subsubsection*{Digital Domain}
Digital Domain is a digital production company, and it stands out from its competitors because it provides a unique service that no one has seen by incorporating the director's creative vision, and  provides quick customer service responses.
\subsubsection*{Lean Manufacturing's Next Life}
\begin{itemize}
\item Competition has forced businesses to increase their efficiency to reduce expenses... leads to lean manufacturing.
\item Successful companies are able to improve customer response rate, while improving quality and productivity
\item Become more responsive to customers while reducing level of inventory.
\item Companies are increasing sales while reducing size of facilities.
\end{itemize}
\section{Finance}
\subsection{Role of Finance}
\begin{defn}
\textbf{Finance} is the function in a business that acquires and manages funds.
\end{defn}
\begin{defn}
\textbf{Financial management} is managing resources to meet goals and objectives.
\end{defn}
\begin{defn}
\textbf{Financial managers} utilizes the data prepared by accountants to make decisions.
\end{defn}
\subsubsection*{Common ways for Financial Failure}
\begin{itemize}
\item \textbf{Undercapitalization} - not enough funds
\item \textbf{Cash flow} - Even if a company is making profit, a company may be unable to pay debt in time due to poor cash flow.
\item \textbf{Expense Control} - Expenses are too high compared to revenue.
\end{itemize}
\begin{defn}
A \textbf{short-term forecast} predicts revenues and expenses for a period of one year or less while \textbf{long-term forecast} predicts it for a period longer than a year (up to five to ten years).
\end{defn}
\begin{defn}
A \textbf{budget} uses management's forecasts to allocate the use of specific resources in a company.
\end{defn}
\begin{defn}
\textbf{Operating (master) budget} encompasses all of the firm's other budgets and summarizes the business's proposed financial activities..
\end{defn}
\begin{defn}
\textbf{Capital budget} highlights a firm's spendng plans on capital expenditures.
\end{defn}
\begin{defn}
\textbf{Cash budget} estimates cash flow.
\end{defn}
\begin{defn}
\textbf{Financial control} is the process where forecasts are compared with actual results.
\end{defn}
\subsection{Need for Funds}
A few reasons why a company may need funding include:
\begin{itemize}
\item \textbf{Day-to-day needs} - Salaries, current liabilities. Always pay last minute due to time value of money.
\item \textbf{Credit operations} - Customers may defer payment and company may not be able to pay off its debt. Must offer incentives for customers to pay earlier (discounts) and maintain a steady cash flow.
\item \textbf{Inventory} - Having enough funds to purchase inventory to meet sales (sometimes before previous sales are collected).
\item \textbf{Capital Expenditures} - Investment in long-term assets or intangible assets.
\end{itemize}
\subsection{Obtaining Short-Term Financing}
Sources to obtain short-term funding include:
\begin{itemize}
\item \textbf{Trade credit} - Buy now, pay later. Discount usually included for earlier payment.
\item \textbf{Family \& Friends} - Companies might waste this money due to lack of understanding of risks.
\item \textbf{Commercial banks} - Able to assess risks and needs. Bank may help in fixing cash flow issues.
\item \textbf{Factoring} - Selling accounts receivable for cash (at a discount).
\item \textbf{Commercial paper} - Unsecured promissory notes valued above \$$100,000$ and due in less than one year. Lower interest rates than conventional loans and usually only applicable to large firms with high credit reputations.
\item \textbf{Credit cards} - utilize personal line of credit.
\end{itemize}
\begin{defn}
A \textbf{promissory note} is a written contract with a promise to pay. These can be sold to banks.
\end{defn}
\begin{defn}
A \textbf{secured loan} is backed up by collateral. (Item lender gets if borrower is unable to pay back)
\end{defn}
\begin{defn}
An \textbf{unsecured loan} does not require collateral
\end{defn}
\begin{defn}
A \textbf{line of credit} is the amount of unsecured short-term-funding a bank will lend to the business.
\end{defn}
\begin{defn}
A \textbf{revolving credit agreement} is a line of credit that is guaranteed but may have a fee.
\end{defn}
\begin{defn}
\textbf{Commercial finance companies} provide short-term loans if tangible assets are offered as collateral. (Higher interest rate due to higher risk)
\end{defn}
\subsection{Obtaining Long-Term Financing}
\subsubsection*{Debt Financing}
\begin{defn}
A \textbf{term-loan agreement} is a promissory note that requires borrower to repay loan in specific installments. (Interest is tax deductible)
\end{defn}
\begin{defn}
\textbf{Risk/return tradeoff} is the principle that higher risk requires higher return in the form of interest.
\end{defn}
\begin{defn}
A \textbf{bond} is a long-term legal obligation of to make regular interest payments and repay the principal amount at maturity.
\end{defn}
\begin{defn}
\textbf{Debenture bonds} are not backed up by collateral.
\end{defn}
\begin{defn}
\textbf{Sinking fund} is a reserve account which the issuer of a bond periodically sets aside some money.
\end{defn}
\begin{defn}
\textbf{Callable bonds} may be repaid before maturity (reducing the amount of interest payments).
\end{defn}
\begin{defn}
\textbf{Convertible bonds} may be converted to common shares at the discretion of the holder.
\end{defn}
\subsubsection*{Equity Financing}
\begin{defn}
\textbf{Stocks} represent ownership of a company
\end{defn}
\begin{defn}[IPO]
An \textbf{initial public offering} is the first time corporations offer to sell new stock to the general public.
\end{defn}
\begin{defn}
A \textbf{stock certificate} is evidence of stock ownership.
\end{defn}
\begin{defn}
\textbf{Dividends} are part of a firm's profits that may be distributed to shareholders.
\end{defn}
\begin{defn}
\textbf{Common shares} have voting rights and a right to share the firm's profits through dividends.
\end{defn}
\begin{defn}
\textbf{Preferred shares} have priority in payment of dividends and have a prior claim on a company's assets if it is liquidated. These do not include voting rights. Similar to bonds, preferred shares may be \textbf{callable}, \textbf{convertible}, and \textbf{cumulative}.
\end{defn}
\begin{defn}
\textbf{Venture capital} is money invested in new or emerging companies.
\end{defn}
\begin{defn}
\textbf{Leverage} is raising funds through debt to increase the potential return of an investment.
\end{defn}
\subsection*{Further Readings}
\subsubsection*{Interview with Tim Jackson}
\begin{itemize}
\item You will make mistakes
\item Use multiple sources of financing
\item Sometimes it's too early to consider venture financing. Just an idea is not enough
\end{itemize}
\section{Ethics and Social Responsibility}
\begin{defn}
\textbf{Ethics} are the standards of moral behaviour. Basic ethics include:
\begin{enumerate}
\item Integrity, respect for human life, self-control, honesty, courage, self-sacrifice
\item Golden Rule: Do unto others as you would have them do unto you
\end{enumerate}
\end{defn}
\begin{defn}
\textbf{Ethical dilemmas} occur when one must choose between two equally unsatisfactory alternatives. Questions to consider:
\begin{itemize}
\item \textbf{Is it legal?} - Law and company policy
\item \textbf{Is it balanced?} - Fair for all parties?
\item \textbf{How will it make one feel?} - Reaction if others knew about the decision
\end{itemize}
\end{defn}
\subsection{Managing businesses Ethically}
\begin{defn}
\textit{"Ethics is caught more than it is taught"} - People learn from observing what others do, not from what they say.
\end{defn}
\begin{exmp}
Managers may indirectly \textbf{encourage} unethical behaviour by offering poorly designed incentive programs.
\end{exmp}
\begin{defn}
\textbf{Compliance based ethics codes} - Forced ethics\\
Emphasizes preventing unlawful behaviour by increasing control and penalize wrondoers.
\end{defn}
\begin{defn}
\textbf{Integrity-based ethic codes} - Guided ethics.\\
Define organization's guiding values, and creates an environment that supports ethical behaviour.
\end{defn}
\begin{defn}
A \textbf{whistleblower} is someone who reports illegal or unethical behaviour.
\end{defn}
Steps to improve business ethics:
\begin{itemize}
\item Top management must adopt and support the code of conduct.
\item Employees must understand that ethical behaviour requirement starts at the top.
\item managers must be trained to consider ethical implications of business decisions.
\item Ethics office must be set up for reporting unethical behaviour. Protect the whistleblower.
\item External stakeholders should be informed of ethics program.
\item Ethics code must be enforced (punishments).
\end{itemize}
\begin{defn}
An \textbf{ethics officer} (as a counselor or investigator) sets a positive tone, and communicates and relates well to employees at all levels to ensure that proper ethics are followed.
\end{defn}
\begin{defn}[SOX]
The \textbf{Sarbanes-Oxley Act} was created to ensure that a public corporation's financial information was accurate and reliable.
\end{defn}
\begin{defn}[FAA]
\textbf{Federal Accountability Act} was created to make the federal government more accountable and increase transparency in government operations.
\end{defn}
\subsection{Corporate Social Responsibility}
\begin{defn}[CSR] 
\textbf{Corporate social responsibility} is the concern businesses have for the welfare of society. Social responsibility dimensions include:
\begin{itemize}
\item \textbf{Philanthropy} - Charitable donations
\item \textbf{Social Initiatives} - Using company resources for a good cause
\item \textbf{Responsibility} - No discrimination, safe products, safe working environment, etc
\item \textbf{Policy} - Position a firm takes on sicla and political issues.
\end{itemize}
\end{defn}
\subsubsection*{Views on Corporate Responsibility}
\begin{itemize}
\item \textbf{Strategic Approach} - Make money for shareholders
\item \textbf{Pluralist Approach} - Recognize special responsibility of management to optimize profits.
\end{itemize}
\subsubsection*{Responsibility to Customers}
\begin{itemize}
\item Rights to safety, information, choice, and opinion.
\item Honesty
\item Build trust through credibility
\end{itemize}
\subsubsection*{Responsibility to Investors}
Unresolved discussion: Which of the following methods is correct:
\begin{itemize}
\item Companies need profits to undertake social responsibility initiatives
\item By being socially responsible, a company will benefit financially
\end{itemize}
\begin{defn}
\textbf{Insider trading} takes advantage of private company information (not known to the public) to have an edge while making investment decisions Insider trading is illegal.
\end{defn}
\subsubsection*{Responsibility to Employees}
\begin{itemize}
\item Create jobs, fairly reward employees, and allow advancement options
\item Mutual respect
\item Happy employees perform better and help the company generate more revenue
\end{itemize}
\subsubsection*{Responsibility to Society}
\begin{itemize}
\item Create wealth to distribute to stakeholders
\item Corporate social responsibility
\end{itemize}
\subsubsection*{Responsibility to the Environment}
\begin{itemize}
\item Protect environment, increased costs but can be charged to customers
\item Carbon credits
\end{itemize}
\subsection{Social Auditing}
\begin{defn}
\textbf{Social audit} is a systematic evaluation of an organization's progress to implementing programs that are socially responsible.
\end{defn}
\begin{defn}
\textbf{Triple bottom line: People, Planet, Profit} is a framework for measuring and reporting corporate performance against economic, social, and environmental parameters.
\end{defn}
\begin{defn}
\textbf{Sustainable development} is the implementation of a process that integrates environmental, economical, and social considerations into decision making.
\end{defn}
\begin{defn}
\textbf{Fair trade} is a policy to ensure producers in developing countries are paid a fair price for the goods we consume.
\end{defn}
\subsection*{Further Reading}
\subsubsection*{Business Ethics}
An example of negative ethics is saying that everything stated is permitted. Guidelines for ethical behaviour include positive self-esteem, empower employees, and set a positive culture.

\subsubsection*{Sustainability with 360 Energy}
Significant productivity improvements come from having employees who care about sustainability. Ways to keep a sustainability momentum going: 
\begin{itemize}
\item Annual budgeting process includes component dealing with sustainability initiatives
\item Recognition and reward system to acknowledge employees who contribute to sustainability
\item Monthly accountability meetings will have an ageda item dealing with sustainability
\end{itemize}
\section{Leadership}
\begin{defn}
\textbf{Resources} incorporates human, natural, and financial resources.
\end{defn}
\begin{itemize}
\item \textbf{Before:} Managers were bosses and ordered people around.
\item \textbf{Now:} Managers serve as coaches, and employees are starting to know more about their tasks than the manager.
\item \textbf{Future:} management will demand a new kind of person: a skilled communicator and team player.
\end{itemize}
\subsection{Functions of Management}
\begin{defn}
\textbf{Management} is the process used to accomplish organizational goals.
\end{defn}
\begin{defn}
\textbf{Planning} is a management function that includes anticipating trends and determining the best strategies and tactics to achieve a goal.
\end{defn}
\begin{defn}
\textbf{Organizing} is a management function that includes designing the structure of the organization and creating a system in which everything flows in harmony.
\end{defn}
\begin{defn}
\textbf{Leading} involves creating a vision, and guiding, training, coaching, and motivating others to achieve organizational goals.
\end{defn}
\begin{defn}
\textbf{Controlling} involves establishing standards to determine whether the organization is progressing towards its goals. Rewards and corrective actions may also occur.
\end{defn}
\subsection{Planning}
\begin{defn}
A \textbf{vision} is an encompassing explanation of why the organization exists and where it's going.
\end{defn}
\begin{defn}
\textbf{Values} are a set of fundamental beliefs that guide a business.
\end{defn}
\begin{defn}
A \textbf{mission statement} is an outline of the organization's fundamental purpose.
\end{defn}
\begin{defn}
\textbf{Objectives} are short-term statements detailing how to achieve organizational goals.
\end{defn}
\begin{defn}
\textbf{SWOT analysis} is a planning tool used to analyze an organizational's strengths, weaknesses, opportunities and threats.
\end{defn}
\begin{exmp}
\textbf{PRIMOF} (Aspects of internal strengths \& weaknesses) - people, resources, ideas, marketing, operations and finance.
\end{exmp}
\begin{defn}
\textbf{Strategic planning} is the process of determining goals of the organization and strategies for obtaining and using resources to achieve them.
\end{defn}
\begin{defn}
\textbf{Tactical planning} is the process of determining detailed, short-term statements to achieve an tactical objective. Typically, this is the responsibility of the lower levels of management.
\end{defn}
\begin{defn}
\textbf{Operations planning} is the process of setting standards and schedules necessary to implement the company's tactical objectives.
\end{defn}
\begin{defn}
\textbf{Contingency planning} is the process of preparing alternative actions in case the primary plans fail.
\end{defn}
\begin{defn}
\textbf{Crisis planning} is reacting to sudden changes in the environment.
\end{defn}
\subsection{Decision Making}
\begin{defn}
\textbf{Rational decision-making model} - Define situation, describe and collect information, find alternatives, reach consensus, decide on the best, implementation, decision evaluation.
\end{defn}
\begin{defn}
\textbf{Problem solving} is a less formal version of the decision-making process.
\end{defn}
\begin{defn}
\textbf{Brainstorming} consists of coming up with ideas.
\end{defn}
\begin{defn}
\textbf{PMI} is listing the positives, minuses, and interesting for a solution.
\end{defn}
\subsection{Organization System}
\begin{defn}
An \textbf{organization chart} shows relationships among people and divides the organization's work.
\end{defn}
\begin{defn}
\textbf{Top management} is the highest level of management, and consists of the president and other company executives.
\end{defn}
\begin{defn}
\textbf{Middle management} includes managers who are responsible for tactical planning and controlling.
\end{defn}
\begin{defn}
\textbf{Supervisory management} or \textbf{first-line managers} are managers who are directly responsible for supervising workers and evaluating daily performances.
\end{defn}
\subsubsection*{Skills}
\begin{itemize}
\item \textbf{Technical} - Ability to perform tasks in a discipline
\item \textbf{Human relations} - Communication and motivation.
\item \textbf{Conceptual} - Big picture of the organization and relationships among its parts.
\end{itemize}
One of the greatest management challenges going forward is creating a unified system out of multiple organizations.
\subsection{Leading}
\subsubsection*{Characteristics of a good leader}
\begin{itemize}
\item Communicate a vision and rally others around that vision
\item Establish corporate values
\item Promote corporate ethics
\item Embrace beneficial transformational change
\item Stress accountability and responsibility
\end{itemize}
\begin{defn}
\textbf{Transparency} is the presentation of a company's facts in a way that is clear, accessible, and apparent to stakeholders.
\end{defn}
\begin{defn}
\textbf{Autocratic leadership} is making managerial decisions without consulting others
\end{defn}
\begin{defn}
\textbf{Democratic leadership} consists of managers and employees working together to make decisions
\end{defn}
\begin{defn}
\textbf{Laissez-faire} or \textbf{free-rein leadership} involves managing setting objectives and doing whatever they want (baylife).
\end{defn}
\begin{defn}
\textbf{Transformational leadership} occurs when visionary leaders can influence others to follow them in working to achieve a goal.
\end{defn}
\begin{defn}
\textbf{Transactional leadership} is associated with employees who are motivated by a system of reward.
\end{defn}
\begin{defn}
\textbf{Knowledge management} is finding the right information, making it easily accessible, and informing others within the firm.
\end{defn}
\subsection{Controlling}
Controlling procedure:
\begin{itemize}
\item Establish clear performance standards.
\item Monitoring and recording actual performance.
\item Compare with plans and standards
\item Communicate deviations to employees involved
\item Taking corrective action when needed, rinse and repeat.
\end{itemize}
\begin{defn}
\textbf{External customers} are customers who buy a products. \textbf{Internal customers} are individuals and units within the firm that receive services from other individuals or units. (Eg: salespeople using market research)
\end{defn}
\subsection*{Further Reading}
\subsubsection*{The Five Competitive Forces that Shape Strategy}
In Mike's Bikes, the five competitive forces include current competitors, potential competitors, substitute products, suppliers, and customers.
\section{Human Resources Management}
\begin{note}
This section is taken with a different note-taking style than the previous sections. Information may be incomplete.
\end{note}
\begin{defn}
\textbf{Human resource management} involves recruiting, selecting, developing, motivating, evaluating, compensating, and scheduling employees to achieve organizational goals.
\end{defn}
Role of HRM has evolved:
\begin{itemize}
\item Recognition that employees are ultimate resource
\item Changes in law changed traditional practices
\end{itemize}
Talent is key to launching a winning startup. Today, qualified employees are scarcer.
\begin{defn}
\textbf{Underemployed workers} are ones who have more skills that what their current job requires.
\end{defn}
\begin{defn}
\textbf{Job analysis} is a study of what employees do in relation to their job title
\end{defn}
\begin{defn}
\textbf{Job description} is a summary of the objectives, type of work, responsibilities, working conditions, and relationship with other functions.
\end{defn}
\begin{defn}
\textbf{Job specification} is a written summary of the minimal qualifications
\end{defn}

\begin{itemize}
\item Prepare HR inventory - Details such as name, education, capabilities, and training.
\item Job Analysis
\item Forecast HR Supply and Demand
\item Establish strategic plan
\end{itemize}
\begin{defn}
\textbf{Recruitment} is finding the right people at the right time.
\end{defn}
Difficulties:
\begin{itemize}
\item Policies - union contracts, wages, promotion
\item Hiring discrimination
\item Corporate culture
\item Availability of skilled workers - sometimes they must be internally trained
\end{itemize}
\begin{defn}
\textbf{Selection} is finding who should be hired to serve the best interests of both parties.
\end{defn}
\begin{itemize}
\item Obtaining complete application forms
\item Initial and follow-up interviews
\item Employment Tests
\item Background investigations
\item Establish trial periods
\end{itemize}
\begin{defn}
\textbf{Contigent workers} are workers who do not have full-time employment. Cons include little benefits, and less salary.
\end{defn}
\begin{defn}
\textbf{Training} involves improving short term skills while \textbf{developing} focuses on developing long-term abilities.
\end{defn}
\begin{defn}
\textbf{On-the-job-training} is either kinesthetic learning (doing) or shadowing (watching).
\end{defn}
\begin{defn}
\textbf{Apprentice programs} are training programs which involve a learner working with an experienced employee to hopefully one day get on their level.
\end{defn}
\begin{defn}
\textbf{Off-the-job training} involves developing skills to foster personal development. (Time management, stress management, health, nutrition, etc)
\end{defn}
\begin{defn}
\textbf{Vertibule training}, also known as near the job training is done in classrooms where the equipment used is similar to those used on the job.
\end{defn}
\begin{defn}
\textbf{Job simulation} is using equipment that simulates job conditions and tasks. This is used because potential cost for real world mistakes is huge.
\end{defn}
Most management training programs include:
\begin{itemize}
\item On-the-job coaching - Coaching
\item Understudy positions - Working as assistants
\item Job rotation - Learn about different functions of organization
\item Off-the-job training - Schools and seminars
\end{itemize}
\begin{defn}
\textbf{Enabling} is the act of equipping workers with the education and tools needed to make their own decisions.
\end{defn}
\begin{defn}
\textbf{Networking} is establishing and maintaing contacts with key people, and use the contacts to make relationships that serve as informal development systems.
\end{defn}
\begin{defn}
\textbf{Mentors} are experienced employees who guide lower-level employees.
\end{defn}
\begin{defn}
A \textbf{performance appraisal} is an evaluation where the performance level of employees is compared to established standards. This is used for promotions, compensations, additional training, or firing.
\begin{enumerate}
\item Establish Performance Standards
\item Communicate Standards
\item Evaluating Performance
\item Discussing Results
\item Take Corrective Action
\item Use Results for Decision Making
\end{enumerate}
\end{defn}
\begin{defn}
The \textbf{360-degree review} suggests that management should gather opinions from all around the employee, including those under, and on the same level as the employee.
\end{defn}
\subsection{Compensation}
Compensation is one of the main tools that companies use to attract and retain qualified employees, and is one of their largest operating costs. A carefully managed compensation program should:
\begin{itemize}
\item Attract workforce in the right quantity and quality.
\item Provide employees with incentive to work productively
\item Retain employees and reduce their chance of leaving and going to competitors
\item Maintain a competitve position in the marketplace. Low costs, high productivity.
\item Provide employees with financial security. (eg: Insurance \& retirement benefits)
\end{itemize}
\begin{defn}
\textbf{Pay equity} refers to equal pay for work of equal value.
\end{defn}
\begin{defn}
\textbf{Gender wage gap} is the difference between the wages of men and women. Currently, women earn 71 to 75\% of men's earnings, despite having the same level of education. Some possible explanations for this include:
\begin{itemize}
\item Women leaving and re-entering the work force to meet family care-giving responsibilities. Loss of seniority, advancement, and wages.
\item Occupation segregation, undervalued jobs for women. (eg: Child care)
\item Historically, lower levels of education. (Less of a factor nowadays)
\item Less unionization among female workers
\item Discrimination in hiring, promotion, and compensation practices.
\end{itemize}
\end{defn}
\begin{exmp}
Many companies use the Hay pay system. A compensation plan is created based on job tiers, where placement is determined by know-how, problem solving, and accountability.
\end{exmp}
\begin{itemize}
\item \textbf{Salary:} Fixed compensation computed weekly, bi-weekly, or monthly pay periods. No additional pay for extra hours.
\item \textbf{Hourly Wage:} Wage based on number of hours or days worked. Often, employees punch time clock. Used for most blue-collar and clerical workers.
\item \textbf{Piecework System:} Wage based on number of items produced. Creates incentive to work productively.
\item \textbf{Commission Plans:} Pay based on a percentage of sales.
\item \textbf{Bonus Plans:} Extra pay for accomplishing or surpassing objectives. Monetary bonus is pure cash, while cashless bonuses include time off, recognition, etc.
\item \textbf{Profit Sharing Plans:} annual bonuses paid to employees based onn company's profits. Amount for each employee is set at a perdetermined percentage.
\item \textbf{Gain-Sharing Plans:} Annual bonuses for employees who achieve certain goals.
\item \textbf{Cost-of-Living Allowances:} Annual increases in wage based on Consumer Price Index. Commonly found in union contracts.
\item \textbf{Stock Options:} Right to purchase company stock at a speciic price over a specific period of time.
\end{itemize}
\begin{defn}
\textbf{Fringe benefits} are benefits that provide additional compensation to employees beyond the base wage. Examples include pension plans, group insurance, termination pay, company car, recreation facilities, and holiday pay.
\end{defn}
\begin{defn}
\textbf{Soft benefits} consist of benefits that help workers maintain the balance between work and family life. Perks include on-site haircuts, concierge services, and free food.
\end{defn}
\begin{defn}
\textbf{Cafeteria-style benefits plans} are flexible benefits plans where the employee can choose the benefits they want up to a certain dollar amount.
\end{defn}
\subsection{Scheduling Employees}
\begin{defn}
A \textbf{flextime plan} gives employees freedom to choose when to work, as long as they work the required amount of numbers. A disadvantage of this is that managers are required to work longer days to assist and supervise employees.
\end{defn}
\begin{defn}
\textbf{Core time} is the period when all employees are expected to be at their job stations.
\end{defn}
\begin{defn}
\textbf{Compressed workweek} allows employes to work a full number of hours in less than the standard number of days, and take a day off.
\end{defn}
\begin{defn}
\textbf{Telework} occurs when employees can carry out all, or part of their work away from their normal workplace. (Eg: Work from home)
\end{defn}
\begin{defn}
\textbf{Job sharing} is an arrangement where two part-time employees share one full-time job.
\end{defn}
\begin{defn}
When an employee chooses to leave, an outside expert can conduct an \textbf{exit interview} to determine their reasons for leaving, and avoid awkward supervisor/employee interviews that may be dishonest.
\end{defn}
\begin{defn}
\textbf{Employee turnover rate} is the percentage of employees that leave the firm every year.
\end{defn}
\begin{defn}
\textbf{Reverse discrimination} is the unfairness that an unprotected group (eg white males) perceives when protected groups receive prference in hiring and promotion.
\end{defn}
\subsection*{Further Reading}
\subsubsection*{The Real Crisis, We Stopped Being Wise}
The most important skill for any position is people skills. One will be able to relate to each retailer,and work with operations and finance staff. 
\end{document}


