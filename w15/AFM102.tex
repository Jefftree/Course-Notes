\documentclass[english, 12pt]{article}
\usepackage{yingconfig}

\usepackage{booktabs}% http://ctan.org/pkg/booktabs
\newcommand{\tabitem}{~~\llap{\textbullet}~~}
% ========================Variables======================================
\newcommand{\coursecode}{AFM 102}
\newcommand{\coursename}{Managerial Accounting}
\newcommand{\thisprof}{Professor J.J. Tian}
\newcommand{\curterm}{Winter 2014}

\begin{document}
\notesheader
\section{Introduction}
\begin{defn}
\textbf{Managerial accounting} is about providing information internally to managers to meet their decision needs.
\end{defn}

\section{Core Concepts}
\begin{itemize}
\item Cost management
\item Performance measurement and evaluation
\item Budgets
\end{itemize}

\begin{tabular}{|c | p{5cm} | p{5cm}|}
\hline
Qualities & Financial Accounting & Managerial Accounting\\
\hline
Reports & Externally & Internally \\
\hline
Emphasizes & 
\compress
\begin{itemize}[noitemsep]
\item Past activities 
\item Reliability
\item Summary data
\end{itemize}
&
\compress
\begin{itemize}[noitemsep]
\item Future activities
\item Relevance
\item Detailed segmented data
\end{itemize}
\\
\hline
Reporting rules & IFRS (mandatory) & Not mandatory, no rules (driven by decision needs)\\
\hline
\end{tabular}
\begin{qte}
Final exam is not cumulative. Does not included midterm material
\end{qte}
\begin{defn}
\textbf{Decentralization} is the delegation of decising making to managers and providing them with the authority to make key decisions relating to their area of responsibility
\end{defn}

\begin{defn}
\textbf{Product costs} provide future benefits.
\begin{itemize}
\item \textbf{Direct Materials} - Materials that go into the finished product and can be traced to it.
\item \textbf{Direct Labour} - Labour costs that can be traced to individual units of production.
\item \textbf{Manufacturing Overhead} - All other costs
\begin{itemize}
\item \textbf{Indirect labour} - Maintenance, security
\item \textbf{Indirect materials} - Lubricants, cleaning supplies
\end{itemize}
\end{itemize}
\end{defn}
\begin{defn}
\textbf{Period costs} provide benefits that do not carry into the future.
\end{defn}
\[ \text{Beginning Inventory} + \text{Purchases} = \text{Ending Inventory} + \text{COGS}\]


\subsection{Some other stuff}
If people work overtime, only the overtime premium goes into manufacturing overhead. Rest goes into direct labour.
\begin{note}
Idle time goes into manufacturing overhead
\end{note}
\begin{defn}
\textbf{Fixed costs} are costs that remain inchanged within a relevant range. Cost per unit is variable and activity cost is constant.
\begin{itemize}
\item Commmitted: Particularly difficult to adjust, long term: plant, major equipment
\item Discretionary: More flexible, short -term (eg: R \&D, advertising)
\end{itemize}

\end{defn}


\begin{defn}
\textbf{Variable costs} are  costs that change in direct proportion to changes in the level of activity. Cost per unit is constant and activity cost is variable.
\end{defn}

\begin{defn}
\textbf{Opportunity cost} is the potential benefit tht is given u when an alternative is selected over another. Not recorded, but should be considered in every managerial decision.
\end{defn}

\begin{defn}
\textbf{Sunk costs} are costs that have already been incurred and cannot be changed by any decision.
\end{defn}


\begin{defn}
\textbf{Step-variable cost} are costs that are incurred in chunks. Discrete graph increasing in steps.
\end{defn}
\begin{defn}
\textbf{Mixed costs} include both fixed and variable costs. $y = a + bx$.
\end{defn}

\begin{defn}
\textbf{High-low method} uses both the highest activity level and lowest activity level to generate a variable cost per unit (slope). Then use point-slope graph formula to calculate fixed cost.
\[\f{\text{\$ at highest activity level } - \text{ \$ at lowest activity level}}{\text{Highest activity }-\text{ Lowest activity}}\]
\end{defn}

\begin{defn}
\textbf{Regression analysis} utilizes all teh available data points, and tries to fit a line to the data points while attempting to minimize errors. 
\end{defn}

\begin{defn}
\textbf{The Contribution Format} is an income statement format that separates expenses into fixed costs and variable costs. Subtotals vary but total cost is the same.
\end{defn}

\begin{defn}
\textbf{Contributon margin} is the amount remaining from sales after variable costs are deducted.
\end{defn}
\end{document}

