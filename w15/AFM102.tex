\documentclass[english, 12pt]{article}
\usepackage{yingconfig}

\usepackage{booktabs}% http://ctan.org/pkg/booktabs
\newcommand{\tabitem}{~~\llap{\textbullet}~~}
% ========================Variables======================================
\newcommand{\coursecode}{AFM 102}
\newcommand{\coursename}{Managerial Accounting}
\newcommand{\thisprof}{Professor J.J. Tian}
\newcommand{\curterm}{Winter 2014}

\begin{document}
\notesheader
\section{Introduction}
\begin{defn}
\textbf{Managerial accounting} is about providing information internally to managers to meet their decision needs.
\end{defn}

\begin{itemize}
\item Cost management
\item Performance measurement and evaluation
\item Budgets
\end{itemize}

\begin{tabular}{|c | p{5cm} | p{5cm}|}
\hline
Qualities & Financial Accounting & Managerial Accounting\\
\hline
Reports & Externally & Internally \\
\hline
Emphasizes &
\compress
\begin{itemize}[noitemsep]
\item Past activities
\item Reliability
\item Summary data
\end{itemize}
&
\compress
\begin{itemize}[noitemsep]
\item Future activities
\item Relevance
\item Detailed segmented data
\end{itemize}
\\
\hline
Reporting rules & IFRS (mandatory) & Not mandatory, no rules (driven by decision needs)\\
\hline
\end{tabular}
\begin{qte}
Final exam is not cumulative. Does not included midterm material
\end{qte}
\begin{defn}
\textbf{Decentralization} is the delegation of decising making to managers and providing them with the authority to make key decisions relating to their area of responsibility
\end{defn}

\section{Cost Classifications}

\begin{defn}
\textbf{Product costs} provide future benefits.
\begin{itemize}
\item \textbf{Direct Materials} - Materials that go into the finished product and can be traced to it.
\item \textbf{Direct Labour} - Labour costs that can be traced to individual units of production.
\item \textbf{Manufacturing Overhead} - All other costs
\begin{itemize}
\item \textbf{Indirect labour} - Maintenance, security
\item \textbf{Indirect materials} - Lubricants, cleaning supplies
\end{itemize}
\end{itemize}
\end{defn}
\begin{defn}
\textbf{Period costs} provide benefits that do not carry into the future.
\end{defn}

\subsubsection*{Equations}
\[ \text{Beg RM (raw materials)} + \text{Purchases} - \text{RM Used} = \text{Endng RM} \]
\[\text{RM} + \text{Direct Labour} + \text{MOH} = \text{Total Manufacturing Costs}\]
\[\text{ Beg WIP} + \text{Total Manufacturing Costs} - \text{Ending WIP} = \text{COGM}\]
\[ \text{Beginning Inventory} + \text{Purchases} = \text{Ending Inventory} + \text{COGS}\]

\begin{note}
If people work overtime, only the overtime premium goes into manufacturing overhead. Rest goes into direct labour. Idle time is added to MOH.
\end{note}

\begin{defn}
\textbf{Fixed costs} are costs that remain unchanged within a relevant range. Cost per unit is variable and activity cost is constant.
\begin{itemize}
\item Commmitted: Particularly difficult to adjust, long term: plant, major equipment
\item Discretionary: More flexible, short-term (eg: R \&D, advertising)
\end{itemize}
\end{defn}

\begin{defn}
\textbf{Variable costs} are costs that change in direct proportion to changes in the level of activity. Cost per unit is constant and activity cost is variable.
\end{defn}

\begin{defn}
\textbf{Differential Costs} are the differences in cost between any two alternatives.
\end{defn}

\begin{defn}
\textbf{Opportunity cost} is the potential benefit that is given up when an alternative is selected over another. Not recorded, but should be considered in every managerial decision.
\end{defn}

\begin{defn}
\textbf{Sunk costs} are costs that have already been incurred and cannot be changed by any decision.
\end{defn}

\section{Cost Behaviour}

\begin{defn}
\textbf{Step-variable cost} are costs that are incurred in chunks. Discrete graph increasing in steps.
\end{defn}

\begin{defn}
\textbf{Mixed costs} include both fixed and variable costs. $y = a + bx$.
\end{defn}

\begin{defn}
\textbf{High-low method} uses both the highest activity level and lowest activity level to generate a variable cost per unit (slope). Then use point-slope graph formula to calculate fixed cost. x-axis would be activity levels, and y-axis would represent costs.
\[\f{\text{\$ at highest activity level } - \text{ \$ at lowest activity level}}{\text{Highest activity }-\text{ Lowest activity}}\]
\end{defn}

\begin{defn}
\textbf{Regression analysis} utilizes all the available data points, and tries to fit a line to the data points while attempting to minimize errors.
\end{defn}

\begin{defn}
\textbf{The Contribution Format} is an income statement format that separates expenses into fixed costs and variable costs. Subtotals vary but total cost is the same.
\end{defn}

\begin{defn}
\textbf{Contribution margin} is the amount remaining from sales after variable costs are deducted.
\end{defn}

\section{Cost Volume Profit}

\begin{defn}
The \textbf{break-even point} is where contribution margin is equal to the fixed cost.
\end{defn}

\begin{defn}
The \textbf{Contribution Margin ratio} represents that for every \$1 increase of sales, CM will increase by the ratio.
\[ \f{\text{Total CM}}{\text{Total Sales}}\]
\end{defn}

\subsubsection*{Break-even Analysis}
\begin{mthd}
\textbf{Equation Method:}
\[ \text{Sales} \times x - \text{VC} \times x = \text{FC}\]
\end{mthd}
\begin{mthd}
\textbf{Contribution Margin Method:}
\[ \text{Break-even units} = \f{\text{Fixed Expenses}}{\text{Unit CM}}\]
\[ \text{Break-even sales} = \f{\text{Fixed expenses}}{\text{CM Ratio}}\]
\end{mthd}

\begin{defn}
\textbf{Margin of safety} is the excess of budgeted/actual sales over the break-even volume of sales. Amount that sales can drop before incurring loss.
\end{defn}

\begin{qte}
Be proficient with calculations involving break-even analysis using contribution margin.
\end{qte}

\begin{defn}
\textbf{Degree of operating leverage} is $\f{\text{CM}}{\text{Net Income}}$. It measures the sensitivity of operating income to changes in sales.
\begin{itemize}
\item Higher leverage: small changes in sales leads to large shift in income.
\item Greatest at break-even point
\item Higher leverage if more FC than VC.
\end{itemize}
\end{defn}

\begin{mthd}
To find the \textbf{indifference point} between two costs, find the equations of both lines in $y = ax + b$ form, and then solve the system of linear equations.
\end{mthd}
\begin{exmp}
Find the equation given this information
\begin{center}
\begin{tabular}{|l|l|}
\hline
Item & Cost \\
\hline
DM Per Unit & $9.00$ \\
DL per unit & 1.2 \text{DLH} at 14/\text{DLH} \\
Variable OH per unit & $0.75$ of DL costs \\
Fixed Man Costs & $1,108,000$ \\
Fixed S \& A & $1,685,000$ \\
Selling price / unit & $60$ \\
Variable selling cost/unit & $4$ \\
\hline
\end{tabular}
\end{center}
\begin{align*}
 \text{VC/unit} &= 9 + 14 \times 1.2 + 14 \times 1.2 \times 0.75 + 4 = 42.4 \\
\text{FC} &= 1108000 + 1685000 = 2973000 \\
\text{Equation: } y &= 42.4x + 2793000
\end{align*}
\end{exmp}

\begin{defn}
\textbf{Sales mix} is the mix of different types of products that a company sells. These various types are expressed as a percentage of total sales, each having their own variable cost.
\end{defn}
\todo{Go over this one more time}

\section{Job-Order Costing}
\begin{defn}
\textbf{Process Costing} occurs when there is a single homogeneous product (eg: oil, pencil), and each unit is indistinguishable from another.
\[ \text{Unit product cost} = \f{\text{Total Manufacturing Costs}}{\text{Total Units Produced}}\]
\end{defn}

\begin{defn}
\textbf{Job-order costing} occurs when there are many different products or services (eg: clothing, renovation). Each unit or batch is distinguishable from another. This is more annoying to calculate.
\end{defn}

\begin{mthd}
\textbf{Allocation base} is an activity or cost driver that is used to assign overhad costs. Labour hours or costs are the most commonly used activity.
\begin{enumerate}
\item Estimate total MOH and allocation base. Then calculate predetermined overhead rate.
\[\text{Predetermined overhead rate} = \f{\text{Estimated total MOH}}{\text{Estimated total units of allocation base}}\]
\item Then calculate the MOH to be allocated to the job
\[ \text{Overhead applied} = \text{Predetermined OH Rate} \times \text{Actual allocation base used by the job}\]
\item Allocate to job
\end{enumerate}
\end{mthd}

\begin{note}
Actual MOH is debited to MOH while estimated MOH is credited. Since OH allocation is based on estimates, it is very likely that actual OH is different from the allocation.
\begin{itemize}
\item \textbf{Underapplied:} Debit balance in MOH, actual > applied.
\item \textbf{Overapplied:} Credit balance in MOH, actual < applied.
\end{itemize}
If the misapplication is not material, just close the balance to COGS. If not, must allocate proportionally to WIP, FG and COGS.
\end{note}

\begin{mthd}
To allocate MOH to WIP, FG, and COGS, must follow these steps:
\begin{enumerate}
\item Determine current allocation of MOH in WIP, FG, and COGS.
\item Calculate \%
\item Allocate the remaining MOH using the same percentages
\item Sum to determine new total allocation and close via transactions.
\end{enumerate}
\end{mthd}

\section{Activity Based Costing}

\begin{defn}
\textbf{Activity based costing} is a costing method based on activities esigned to help managers with internal decisions.
\end{defn}

\begin{defn}
An \textbf{activity} is an event that causes the consumption of OH costs.
\end{defn}

\begin{defn}
An \textbf{activity cost pool} is a grouping of OH costs that relate to a single activity measure.
\end{defn}

\begin{defn}
An \textbf{activity measure} or \textbf{cost driver} is a measure of the use of an activity.
\end{defn}

\begin{qte}
By using ABC,
\begin{itemize}
\item Both manufacturing and non-manufacturing (period) costs may be assigned to products.
\item Some manufacturing costs may be excluded
\item Various overhead cost pools may be used
\item OH rates may be based on capacity rather than estimated level of activity
\end{itemize}
\end{qte}

\begin{mthd}
\textbf{Activity based costing system:}
\begin{enumerate}
\item Identify and define activities, activity cost pools, and activity measures.
\begin{itemize}
\item \textbf{Unit level:} Consumes overhead each time unit is produced.
\item \textbf{Batch-level:} Consumes overhead when batch is produced. Eg: Purchase order, setting up equipment
\item \textbf{Product-level:} Consumes overhead based on specific products. Eg: Product advertising, designing a new product
\item \textbf{Customer-level:} Consumes overhead relating to specific groups of customers. Eg: Internet sales, catalogue mailing.
\item \textbf{Organization-sustaining level:} Consumes overhead regardless of other factors. Eg: Heating, banking fees, preparation of financial statements.
\end{itemize}
\item Assign overhead costs to activity cost pools.
\begin{itemize}
\item Percentage of available capacity consumed by each activity
\end{itemize}
\item Calculate activity rates for each activity
\[\f{\text{Total Activity Cost}}{\text{Total Activity}}\]
\item Assign overhead costs to cost objects using activity rates and measures.
\end{enumerate}
\end{mthd}

\section{Variable Costing}

\begin{defn}
\textbf{Variable costing} is a costing method that includes only variable manufacturing costs in the inventory. This means fixed MOH would be a period cost. Lean manufacturing can reduce the disparity between variable and absorption costing.
\end{defn}

\begin{tabular}{|l|l|l|}
\hline
Production vs Sales & Effects & ABC vs Variable Income \\
\hline
Production > Sales & Fixed MOH is deferred to ending inventory & Absorption > Variable \\
Production = Sales & No effect & Absorption = Variable \\
Production < Sales & Fixed MOH released from beginning inventory & Absorption < Variable\\
\hline
\end{tabular}

\todo{Go over reconcilliation again}

\begin{note}
Advantages and Disadvantages of variable costing:
\begin{itemize}
\item \textbf{Advantages:}
\begin{itemize}
\item Fixed costs are total amounts, increasing opportunity for more effective control over the costs
\item Profits vary directly with sales volume, and not affected by changes in inventory
\end{itemize}
\item \textbf{Disadvantages:}
\begin{itemize}
\item Not acceptable for external reporting
\item difficult to determine fixed vs variable costs
\end{itemize}
\end{itemize}
\end{note}



\end{document}

