\documentclass[english, 12pt]{article}
\usepackage{yingconfig}

% ========================Variables======================================
\newcommand{\coursecode}{CS 136}
\newcommand{\coursename}{.rkt in C}
\newcommand{\thisprof}{Professor M. Petrick}
\newcommand{\curterm}{Winter 2014}

\begin{document}
\notesheader

\section{Modularization}
\begin{defn}
A \textbf{module} is a collection of functions that share a common aspect or purpose. \textbf{Modularization} is dividing programs into modules.
\begin{itemize}
\item Reusability
\item Maintainability
\item Abstraction
\end{itemize}
\end{defn}
\begin{defn}
\textbf{provide} is used in a module to specify the identifiers availabie in the module.
\end{defn}
fun.rkt
\begin{lstlisting}[language=Scheme]
(provide fun?) ;Allows use of function outside of program
(define lofn `(-3 7 42 136 1337 4010 8675309))
;; (fun? n) determines if n is a fun integer
;; fun?: Int -> Bool
(define (fun? n)
  (not (false? (member n lofn))))
\end{lstlisting}
\begin{defn}
\textbf{require} is used to identify a module that the current program depends on.
\end{defn}
implementation.rkt
\begin{lstlisting}[language=Scheme]
(require "fun.rkt")
;;Able to use provided functions in required file
(fun? 7) ; => #t
(fun? -7) ; => #f
\end{lstlisting}
\subsection{Scope}
\begin{itemize}
\item \textbf{Local:} Visible only in local region
\item \textbf{Module:} Only visible in the module it is defined in
\item \textbf{Program:} Visible outside the module.
\end{itemize}
\begin{qte}
\textbf{require} also outputs the final value of any of the top-level expressions in the module. Only definitions should be included in modules.
\end{qte}
\begin{defn}
A module \textbf{interface} is the list of functions that a module provides. Documentation should be provided.
\begin{itemize}
\item Description of module
\item List of functions provided
\item Contract and purpose for each provided function
\end{itemize}
\end{defn}
\begin{defn}
The \textbf{implementation} is the code for the module.
\begin{itemize}
\item Hides implementation details from client
\item Security
\item Flexibility to modify implementation
\end{itemize}
\end{defn}

\begin{defn}
\textbf{High cohesion} means that all interface functions are related.
\end{defn}
\begin{defn}
\textbf{Low coupling} means that there is little interaction between modules.
\end{defn}

\begin{qte}
Always truncate decimals
\end{qte}


\begin{lstlisting}[language=C]
int main (void) {
  printf(``Hello World! \n'')
}
\end{lstlisting}
\begin{defn}
\%d is used as a placeholder to the values that follow.
\begin{lstlisting}[language=C]
printf("%d plus %d is: %d\n", 1 + 1, 2, 2 + 2);
\end{lstlisting}
In racket, ~a is used as a placeholder.
\begin{lstlisting}[language=Scheme]
(printf ``There are ~a lights!\n'' ``four'')
(printf ``There are ~a lights!\n'' 'four) ; Both lines are same
\end{lstlisting}
\end{defn}

\begin{defn}
Structures in C are very similar to racket.
\begin{lstlisting}[language=C]
struct posn {
  int x;
  int y;
}; //Do not forget the semicolon

const struct posn p = {3,4}; // Initialization
const struct posn pp = {y=4,x=3}; // This works too
const struct posn pp = {x=3}; // Uninitialized integers are set to 0.

const int a = p.x;
const int b = p.y;
\end{lstlisting}
\end{defn}


\begin{defn}
\textbf{begin} produces the value of the last exprssion
\begin{lstlisting}[language=Scheme]
(define (mystery)
  (begin ; implicit, this line not needed
    (+ 1 2) ; evaluated, not used
    (+ 2 2))) ;outputs 4
\end{lstlisting}
\end{defn}
\begin{qte}
Anything that is not \#f in Racket is true.
\end{qte}
\section{Imperative Programming}
\begin{defn}
The \textbf{functional programming paradigm} is to only use constant values that never change. Functions produce new values rather than changing existing ones. In functional programming, there are no side effects.
\end{defn}

\begin{defn}
A \textbf{side effect} does more than produce a value it also changes the state of the program. Sometimes used to debug.
\end{defn}

\begin{defn}
In an expression statement, the \textbf{value} of the expression is \textbf{ignored}.
\begin{lstlisting}[language=C]
3 + 4;
\end{lstlisting}
\end{defn}

\begin{defn}
A \textbf{block} $\{\}$, is known as a compound statement, and contains a sequence of statements.  Within a block, \textbf{local scope definitons} can also be included.
\end{defn}

\begin{defn}
\textbf{printf} in C returns an int representing the number of characters printed.
\end{defn}

\begin{defn}
\textbf{Control flow statements} change the flow of a program and the order in which other statements are executed.
\begin{itemize}
\item \textbf{return} statement ends the execution of a function and returns a value.
\item \textbf{if} and \textbf{else} statements execute statements conditionally
\end{itemize}
\end{defn}
\begin{qte}
The defining characteristic of \textbf{imperative programming paradigm} is to \textbf{manipulate state}.
\end{qte}

\begin{defn}
\textbf{State} refers to the value of a data at a moment in time.
\end{defn}

\begin{defn}
When the value of a variable is changed, it is called \textbf{mutation}.
\begin{lstlisting}[language=C]
int x = 5;
struct posn p = {3,4};
\end{lstlisting}
\end{defn}
\begin{defn}
\textbf{Prefix} and \textbf{postfix} increment operator:
\begin{lstlisting}[language=C]
x++ // Produces old value, and increments as side effect
++x // Increments x and then produces the value
\end{lstlisting}
\end{defn}

\section{C Model}

\begin{defn}
A \textbf{bit} has two states: $0$ or $1$. A \textbf{byte} is $8$ bits of storage. Each byte is in one of $256$ possible states.
\end{defn}


\begin{defn}
\textbf{Memory addresses} are represented in hex (prefixed with $0x$), so a typical address would be $0xFFFFF$.
\end{defn}

\begin{defn}
\textbf{sizeof} produces the amount of space (bytes) a variable uses.
\begin{itemize}
\item A \textbf{char} is $1$ byte.
\item An \textbf{int} is $4$ bytes.
\item An \textbf{address} is $8$ bytes.
\end{itemize}
\end{defn}

\begin{note}
When a variable is initialized, three steps occur:
\begin{itemize}
\item Reserves space in memory to store the variable
\item Records the address to the location
\item Store the value of the variable at the address.
\end{itemize}
\begin{center}
\begin{lstlisting}[language=C]
int n = 4;
\end{lstlisting}
\begin{tabular}{|c|c|c|c|}
\hline
identifier & type & bytes & address \\
\hline
$n$ & int & $4$ & $0x5000$ \\
\hline
\end{tabular}
\end{center}
\end{note}
\begin{qte}
A variable definition reserves space, but declaration does not.
\end{qte}

\begin{note}
If an int is larger than the maximum $2^31 - 1$ or smaller than the minimum $- 2^31$, overflow will occur. Remember to always try and avoid chance of overflow wherever possible.
\end{note}

\begin{note}
For characters, A is $65$, $a$ is 97, space is $32$, $0$ is $48$, and newline is $10$ in ASCII.
\end{note}

\begin{note}
The \text{sizeof} a structure is at least the sum of the size of each field.
\end{note}


\begin{defn}
A \textbf{float} represents real numbers and has a larger range than int. Floats are very imprecise, and doubles are usually used instead.
\end{defn}


\subsection{Memory}
Memory can be modelled as
\begin{center}
\begin{tabular}{|c|}
\hline
Code \\
\hline
Read-Only Data \\
\hline
Global Data \\
\hline
Heap \\
\hline
Stack \\
\hline
\end{tabular}
\end{center}
\begin{defn}
Converting source code to machine code is known as \textbf{compiling}
\end{defn}

\begin{note}
Global constants are stored in read-only, and global variables are stored in global data. The space is reserved before execution.
\end{note}

\begin{defn}
\textbf{control flow} is used to model how programs are executed.
\end{defn}

\begin{defn}
The history of what a program needs to do is called the \textbf{call stack}. When a function is called, it is pushed onto the call stack. When a return is used, an entry is popped off the stack.
\end{defn}

\begin{defn}
An entry pushed onto a call stack is a \textbf{stack frame}. A stack frame consists of
\begin{itemize}
\item Argument values
\item Local variables
\item Return address
\end{itemize}
\end{defn}
\begin{qte}
When the function returns, the entire stack frame is destroyed along with its local variables.
\end{qte}

\begin{defn}
When the stack frame is too large, it can collide with other sections of memory. This is called \textbf{stack overflow}.
\end{defn}

\begin{qte}
All global variables that are uninitialized are automatically initialized to 0. Uninitiaized local variables have an arbitrary initial value.
\end{qte}

\subsection{Loops}
\begin{defn}
\textbf{while} reeatedly loops back and executes the statement until the expression is false.
\end{defn}

\begin{defn}
The \textbf{do} statement is similar to the while statement but evaluates the expression after execution. Because of this, the loop is always executed at least once.
\end{defn}

\begin{defn}
\textbf{break} is used to break out of a loop.
\end{defn}

\begin{defn}
\textbf{continue} skips the current block of execution and continues the loop.
\end{defn}

\begin{lstlisting}[language=C]
int num = 6;
while (num!=0) { // 6,3,2,1, end
  if (num == 6) {
    num -= 3;
    continue;
  }
  num --;
}
\end{lstlisting}

\begin{defn}
\textbf{for} is similar to a condensed form of a while loop.
\begin{lstlisting}[language=C]
for (int i = 0 ; i < 5; i++) { body }
\end{lstlisting}
Any component may be omitted in a for loop. An omitted expression is always true. Commas may be used for compound statements in the setup of a for loop.
\end{defn}

\section{Pointers}

\begin{defn}
The \textbf{address operator} \& produces the starting address of where the value of an identifier is stored in memory.
\end{defn}

\begin{defn}
By adding a * before an identifier, it becomes a pointer, and its value is an address.
\begin{lstlisting}[language=C]
i = 42;
int *p = &i; // p points to i
printf(``p is %p'', p); \\ prints the address of i
\end{lstlisting}
\end{defn}

\begin{defn}
The \textbf{indirection operator} * is the inverse of address operator and produces the value of what a pointer points at.
\begin{lstlisting}[language=C]
int = 42;
int *p = &i; //points at address of i
int j = *p // 42
\end{lstlisting}
\end{defn}
\begin{note}
C mostly ignores whitespace, so the following lines are all equivalent.
\begin{lstlisting}[language=C]
int *pi = &i; // style A (preferred)
int * pi = &i; // style B
int* pi = &i; // style C
\end{lstlisting}
\end{note}

\begin{defn}
By adding multiple asterisks, a pointer to a pointer may be declared.
\begin{lstlisting}[language=C]
int i = 42;
int *pi = &i; // address of i
int **ppi = &pi; // address of pi
\end{lstlisting}
\end{defn}

\begin{defn}
\textbf{NULL} is a pointer value that represents that the pointer points to nothing.
\end{defn}

\subsection{Pointer Assignment}
The value of what a pointer is pointing at may be changed. They can be dereferenced to change the value of the variable they point at without actually using the variable.
\begin{lstlisting}[language=C]
int i = 5;
int j = 6;
int *p = &i;
int *q = p;
*q = j; // i = 6
\end{lstlisting}

\begin{note}
Pointers may be used to emulate \textbf{pass by reference} even though C is pass by value.


\begin{lstlisting}[language=C]
void inc(int *p) {
  *p += 1;
}

int main(void) {
  int x = 5;
  inc(&x); // note the &
  printf("x = %d\n", x); // NOW it's 6
}
\end{lstlisting}
This may also be used on structures, but brackets must be added around the dereference \textbf{(*p).x}
\end{note}

\begin{defn}
The \textbf{arrow selection operator} (->) combines the indirection and selection operators. This may only be used with a pointer to a structure.
\begin{lstlisting}[language=C]
int sqr_dist (struct posn *p1, struct posn *p2) {
  const int xdist = p1->x - p2->x;
  const int ydist = p1->y - p2->y;
  return xdist * xdist + ydist * ydist;
}
\end{lstlisting}
\begin{note}
These parameters may also be mutated.
\end{note}
\end{defn}

\section{I/O \& Testing}

\begin{defn}
\textbf{fprintf} has an addition parameter that points to a file. It is similar to printf but prints directly to the file.
\begin{lstlisting}[language=C]
int main(void) {
  FILE * file_ptr;
  file_ptr = fopen("hello.txt","w");   // w for write
  fprintf(file_ptr, "Hello World!\n");
  fclose(file_ptr);
}
\end{lstlisting}
\end{defn}

\begin{defn}
In Racket, \textbf{(read)} is used to get a value from keyboard.
\begin{lstlisting}[language=Scheme]
(define key-inp (read))
\end{lstlisting}
Text is interpreted as symbols unless it is surrounded with double quotes.
\end{defn}

\begin{defn}
In C, \textbf{scanf} is used for keyboard input. scanf returns the number of values successfully read. If there is an error, 0 is returned by scanf.
\begin{lstlisting}[language=C]
int count = scanf(``%d'', &i); // Reads integer, and stores it in i. Count should be 1.
\end{lstlisting}
\end{defn}
\begin{note}
When reading in characters, it may be beneficial to ignore whitespace.
\begin{lstlisting}[language=C]
int count = scanf(``%c'', &c); // May read whitespace
int count2 = scanf (`` %c'', &c); // Skips whitespace
\end{lstlisting}
\end{note}

\section{Arrays and Strings}

\begin{lstlisting}[language=C]
int a[5]; // Valid, size defined
int b[] = {4,8,15,16,23,42}; //Valid size can be computed
int c[]; // Invalid
\end{lstlisting}

\begin{defn}
The \textbf{length} if an array is the number of elements in the array.
\end{defn}

\begin{defn}
The \textbf{size of an array} is the number of bytes it occupies in memory.
\end{defn}

\begin{qte}
C does not explicitly keep track of the array length.
\end{qte}

\begin{lstlisting}[language=C]
int a[] = {2,4}; // a by default points to a[0]
assert(&a == &a[0]);
assert(a == &a);
assert(*a == a[0])
\end{lstlisting}

\begin{note}
An array cannot be mutated. Only its elements can change.
\begin{lstlisting}[language=C]
int a[3] = {0, 0, 0};
int b[3] = {1, 2, 3};
a = b; // INVALID
\end{lstlisting}
\end{note}

\begin{qte}
When passing an array to a function, typically the length of the array is unknown and must be provided as a separate parameter.
\end{qte}

\begin{qte}
In an array, pointers can be subtracted. However, pointer arithmetic is only valid within an array.
\end{qte}

\begin{note}
If there are two pointers $p$ and $q$,
\[p-q = \f{(p-q)}{sizeof(*p)}\]
\[p + i = p + i \times sizeof(*p)\]
$p\lbrack i \rbrack $ is equivalent to $*(p + i)$.
\end{note}

\begin{exmp}
In \textbf{array pointer notation} square brackets are not used, and all array elements are accessed through pointer arithmetic.

\begin{lstlisting}[language=C]
// Pointer notation
int sum_array(const int *a, int len) {
  int sum = 0;
  for (const int *p = a; p < a + len; ++p) {
    sum+=*p;
  }
  return sum;
}

// Square bracket notation
int sum_array(const int a[], int len) {
  int sum = 0;
  for (int i = 0; i < len; ++i) {
    sum+=a[i];
  }
  return sum;
}
\end{lstlisting}
\end{exmp}

\begin{defn}
\textbf{Multi-dimensional data} can be represented by mapping the higher dimensions down to one. That is, to select an element at a specific row and column, you would do $\text{data}\lbrack row \& NUMCOLS + col \rbrack$.
\end{defn}

\begin{defn}
\textbf{Function pointers} store the starting address of a function within the code section. A function pointer in C can only point to a function that already exists.
\begin{lstlisting}[language=C]
int add1(int i) {
  return i + 1;
}

int (*fp)(int) = add1;
// Return value first, then name of function, then its parameters
// Now you can call the function with fp(int i)

// Another is to use it as a parameter
// The below function adds 70 to a number, n
int doSomething(int (*fcn)(int), int n) {
  return fcn(n) + 69;
}
\end{lstlisting}
\end{defn}

\begin{note}
These function pointers are useful for abstract list functions seen in Racket.
\begin{lstlisting}[language=C]
void array_map(int (*f)(int), int a[], int len) {
  for (int i = 0; i < len; ++i) {
    a[i] = f(a[i]);
  }
}
\end{lstlisting}
\end{note}
\subsection{Strings}
\begin{defn}
A \textbf{string} in C is just an array of characters terminated by a null character `\\0'. If a string is initialized as an array, the null terminator is necessary, but if it is initialized with double quotes, then the null terminator is automatically added.
\begin{lstlisting}[language=C]
char a[] = {'c','a','t','\0'};
char b[] = ``cat'' // Equivalent
\end{lstlisting}
\end{defn}

\begin{defn}
C strings used in statements (eg with printf) are known as \textbf{string literals}. For these statements, a null terminated const char array is created in the \textbf{read-only data section}.
\end{defn}

\begin{defn}
The \textbf{strlen} function returns the length of the string, NOT THE LENGTH OF THE ARRAY. It also does not include the null character. It is found in the <string.h> library.
\end{defn}

\begin{defn}
Strings are compared by their \textbf{lexicographical order}. That is, for each character, sort by the ASCII values of the characters. If the end of one string is encountered, it precedes teh other string. The \textbf{strcmp(s1,s2)} function returns 0 if the strings are identical, -1 if s1 < s2, and 1 if s1 > s2.
\end{defn}

\begin{note}
Do not compare strings directly (eg: s1 == s2). This only compares pointers, not the actual content!.
\end{note}

\begin{qte}
When allocating space for a string, DO NOT FORGET THE NULL CHARACTER.
\end{qte}

\begin{defn}
\textbf{strcpy(char * dest, const char *src)} copies the content of the string src to dest. \textbf{strcat(char * dest, const char *src)} appends the content of src to dest. Make sure that array is large enough so the content may be copied without \textbf{buffer overflow}.
\end{defn}

\section{Efficiency}

\begin{defn}
An \textbf{algorithm} is a step-by-step description of how to solve a problem.
\end{defn}

\begin{defn}
\textbf{Time efficiency} is how long an algorithm takes to solve a problem.
\end{defn}

\begin{defn}
\textbf{Space efficiency} is how much space/memory an algorithm requires to solve a problem.
\end{defn}

\begin{defn}
In this course, the running time of a function is a function of $n$, denoted $T(n)$. $n$ is usually the length of the input (array, number size, etc). They are usually measured in the worst case.
\end{defn}

\begin{defn}
\textbf{Big O Notation} showcases the \textbf{order} of a running time. That is, it is the dominant power as $n \to \infty$.
\end{defn}

\begin{exmp}
When adding two orders, the result is the largest of the two orders.\\
$O(log\,n) + O(n) = (n)$ and $O(1) + O(1) = O(1)$.
\end{exmp}

\begin{exmp}
When multiplying two orders, the result is the product of the two orders.\\
$O(log\,n) \times O(n) = O(n\,log\,n)$ and $O(1) \times O(n) = O(n)$.
\end{exmp}

\begin{defn}
\textbf{Simple functions} are functions without recursion or iteration. In C, all operations and O(1), so the running time of a simple function is
\[O(1) + \cdots + O(1) = O(1)\]
\end{defn}

\todo{Should include racket running times slide?}

\begin{defn}
For recursive functions, we analyze the \textbf{recurrence relation}. For now, use a table to determine the runtime.

\[T(n) = O(1) + T(n - k_1) = O(n)\]
\[T(n) = O(n) + T(n - k_1) = O(n^2)\]
\[T(n) = O(1) + T(\f{n}{k_2})\]
\[T(n) = O(1) + k_2 \cdot T(\f{n}{k_2})\]
\[T(n) = O(n) + k_2 \cdot T(\f{n}{k_2})\]
\[T(n) = O(1) + T(n-k_1) + T(n-k_1') = O(2^n)\]
An example of $2^n$ is the recursive fibonacci sequence.
\end{defn}

\begin{mthd}
\textbf{Procedure for recursive functions}:
\begin{enumerate}
\item Identify order of function excluding recursion
\item Determine size of input for next recursive calls
\item Write full recurrence relation, and look up in a table
\end{enumerate}
\end{mthd}

\begin{defn}
\textbf{Iterative analysis} utilizes \textbf{summations} instead of recurrence relations.
\begin{lstlisting}[language=C]
for (i = 1; i <=n; i++) {
  printf("*");
} //O(n) time
\end{lstlisting}
\[T(n) = \sum_{i=1}^n O(1) = \underbrace{O(1) + \cdots + O(1)}_n = n \times o(1) = O(n)\]
\end{defn}

\begin{note}
If a given list is of constant length (not dependant on input size), all operations are O(1).
\begin{lstlisting}[language=C]
for (int i = 0; i < 696969; i++) {
  sum+=a[i];
}
// O(1) linear time because a large number is still a constant.
\end{lstlisting}
\[\sum_{i=1}^{log\,n} O(1) = O(log\,n)\]
\[\sum_{i=1}^{n} O(1) = O(n)\]
\[\sum_{i=1}^{n} O(n) = O(n^2)\]
\[\sum_{i=1}^{n} O(i) = O(n^2)\]
\end{note}

\begin{mthd}
\textbf{Procedure for iteration}
\begin{enumerate}
\item Work from innermst loop to outermost
\item Determine number of iterations in the loop
\item Determine running time per iteration
\item Write summation and simplify expression
\end{enumerate}
\end{mthd}

\begin{note}
When the loop counter changes geometrically, the number of lterations is often logarithmic.
\begin{lstlisting}[language=C]
while (n > 0) {
  // Do something
  n /= 10;
}
\end{lstlisting}
\end{note}

\begin{exmp}
\textbf{Sorting Algorithm:}
\begin{itemize}
\item Insertion Sort: $O(n^2)$
\item Selection Sort: $O(n^2)$
\item Merge Sort: $O(n\,log\,n)$
\item Quick Sort: $O(n^2)$
\end{itemize}
\end{exmp}

\begin{defn}
A function is \textbf{tail recursive} if the recursive call is alwasy the last expression to be evaluated. With tail recursion, the previous stack frame can be \textbf{reused} for the next recursion.
\end{defn}

\section{Dynamic Memory}

\begin{defn}
\textbf{Dynamic memory} is allocated from the \textbf{heap} while the programming is running.
\end{defn}

\begin{defn}
The \textbf{exit(EXIT\_FAILURE)} function stops the program execution. (Useful for debugging where there are fatal errors.).
\end{defn}

\begin{defn}
The \textbf{heap} can be thought of as a big pile of memory that is available. When memory is dynamically borrowed from the heap, it is known as \textbf{allocation}. When the memory is returned, it is called \textbf{deallocation}.
\end{defn}

\begin{defn}
The \textbf{malloc} function obtains memory from the heap.
\begin{lstlisting}[language=C]
// malloc(s) requests s bytes of memory from the heap and returns
// a pointer to it.
// It returns NULL of not enough memory is available
void * malloc (size_t s);
 //Examples
int * pi = malloc(sizeof(int));
struct posn *pp = malloc(sizeof (struct posn));
char * yes = malloc(sizeof(char) * (strlen(s) + 1))
// Must include null terminator for strings
\end{lstlisting}
\end{defn}

\begin{defn}
The \textbf{free} function returns the memory at the given address back to teh heap. The given parameter must e from a previous malloc.
\end{defn}

\begin{defn}
A \textbf{memory leak} occurs when allocated memory is not eventually freed.
\begin{lstlisting}[language=C]
int * ptr;
ptr = malloc(sizeof(int));
ptr = malloc(sizeof(int)); // Cannot free first block of memory
\end{lstlisting}
Once a free occurs, if the address is used again, it will be invalid and may cause unwanted behaviour in the program.
\end{defn}

\begin{defn}
In Racket, memory does not need to be freed because it has a \textbf{garbage collector}. The program automatically returns memory not in use to the heap. Unfortunately this can affect performance.
\end{defn}

\begin{defn}
The \textbf{realloc} function resizes a given array to a specific length. It allocates or deallocates memory depending on the resize.
\begin{lstlisting}[language=C]
void * realloc (void * ptr, size_t s);
\end{lstlisting}
\end{defn}

\begin{note}
When appending items to a dynamically allocated array, it would take too long to call realloc every time an item was added. A popular strategy to fix this is to \textbf{double} the size of the array when it is full. This reduces the time from $O(n^2)$ to $O(n)$ where $n$ is the number of calls to append.\n
The \textbf{amortized} (averaged) run-time of append would then be $O(n)/n = O(1)$.
\end{note}

\section{Linked Data}

\begin{defn}
A \textbf{linked list node} usually contains an item and a link (pointer) to the next node.
\begin{lstlisting}[language=C]
struct llnode {
  int item;
  struct llnode * next;
}
\end{lstlisting}
\end{defn}

\begin{defn}
In the \textbf{wrapper strategy}, we wrap each list so it is inside of another structure.
\begin{lstlisting}[language=C]
struct llist {
  struct llnode * front;
}
\end{lstlisting}
Some common functions would be to create and destroy the list.
\begin{lstlisting}[language=C]
struct llist * create_list() {
  struct llist *lst = malloc(sizeof(struct llst));
  lst->front = NULL;
  return lst;
}

void destroy_list(struct llist *lst) {
  free_list(lst->front);
  free(lst);
}
\end{lstlisting}
Some additional information may be stored in the wrapper to speed up certain processes. This is known as an \textbf{augmentation strategy}. For example, to retrieve the length of the list, $O(n)$ time would be required, but if the length was stored as another variable, $O(1)$ time would be needed. Two common items are length and a pointer to the last item of the list to make adding to front or back $O(1)$.
\end{defn}

\begin{note}
The wrapper approach introduces entirely new ways that the information can be corrupted. For example, what if the length field does not accurately reflect the true length?\n
When information is stored in more than one way, it is susceptible to \textbf{integrity} (consistency) issues.
\end{note}

\begin{defn}
\textbf{BST}:
\begin{lstlisting}[language=C]
struct bstnode {
  int item;
  strict bstnode * left;
  strict bstnode * right;
}
\end{lstlisting}
\end{defn}

\begin{defn}
The \textbf{depth} of a node is the number of nodes from the root to the node, including the root.
\end{defn}

\begin{defn}
The \textbf{height} of a tee is the maximum depth in the tree.
\end{defn}

\begin{note}
The worst case running time of searching for an item in a tree is when the tree is \textbf{unbalanced} and every node in the tree must be visited. $O(n)$. A \textbf{balanced} tree is where the height is $O(log n)$. A \textbf{self-balancing tree} rearranges nodes to ensure that the tree is always balanced.
\end{note}

\subsection{Graphs}
Linked lists and trees can be thought of as special cases of \textbf{graphs}. Graphs link \textbf{nodes} with \textbf{edges}, and may be undirected, directed, allow cycles, or be acyclic.

\section{Abstract Data Types}
Almost every data structure is some combination of the following \textbf{core} data structures.
\begin{itemize}
\item Primitive
\item Structures
\item Arrays
\item Linked Lists
\item Trees
\item Graphs
\end{itemize}
Selecting an appropriate data structure is important in \textbf{program design}. Be sure to consider the following
\begin{itemize}
\item Frequency of adding/removing items
\item Searching
\item Access to a specific index
\item Preservation of order, rearranged?
\item Duplicates allowed?
\end{itemize}

\begin{defn}
\textbf{Sequenced data} is data that must have its sequence preserved. (eg: competition results). \textbf{Unsequenced data} is data that can be permutated. (eg lists of things).
\end{defn}

\begin{tabular}{|l|c|c|}
\hline
Function & Dynamic Array & Linked List \\
\hline
item\_at & $O(n)$ & $O(n)$ \\
search & $O(n)$ & $O(n)$ \\
insert\_at & $O(n)$ & $O(n)$ \\
insert\_front & $O(n)$ & $O(1)$ \\
insert\_back & $O(1)$ & $O(1)$ \\
remove\_at & $O(n)$ & $O(n)$ \\
remove\_front & $O(n)$ & $O(1)$ \\
remove\_back & $O(1)$ & $O(1)$ \\
\hline
\end{tabular}
\begin{exmp}
\textbf{Good Design Decisions:}
\begin{itemize}
\item Array is good if frequent access of elements at specific positions is needed (random access)
\item Linked list is good for sequenced data that frequently adds and removes data
\item Self-balancing BST is good for unsequenced data if search, add \& remove are all frequent
\item Sortd array is good if you rarely add/remove elements, but frequently search for elements and select data in sorted order.
\end{itemize}
\end{exmp}

\begin{defn}
An \textbf{abstract data type} is a mathematical model for storing and access data through \textbf{operations}. ADTs are \textbf{implemented} as data storage modules that only allows access to the data through teh interface functions.
\end{defn}

\begin{defn}
S supports \textbf{opaque structures} through \textbf{incomplete declarations} where a structure is declared without any fields. Only a pointer to a structure may be declared.
\end{defn}

\begin{defn}
The \textbf{typedef} keyword allows you to create your own type from previously existing types.
\begin{lstlisting}[language=C]
typedef int Integer;
typedef int * IntPtr;

Integer i;
IntPtr p = &i;
\end{lstlisting}
This is often used to simplify complex declarations.
\end{defn}

\begin{exmp}
In general when providing an ADT to a client, you would provide
\begin{itemize}
\item Opaque structure definition (sometimes with typedef)
\item Operations: (create, destroy, other useful operations)
\begin{lstlisting}[language=C]
struct account;
typedef struct account * Account;
//Operations
Account create_account(const char * username, const char * password);
void destroy_acount(Account a);
//Other
bool is_correct_password(Account a, const char *password);
\end{lstlisting}
\end{itemize}
\end{exmp}

\begin{defn}
\textbf{Collection ADTs} are designed to store an arbitrary number of items: stack, queue, sequence, dictionary, and set.
\end{defn}

\begin{defn}
The \textbf{stack ADT} is a collection of items that are stacked togther, items are pushed onto and popped from the top of stack. Stacks follow LIFO.\n
Operations: \textbf{push(s,i), pop(s), top(s), is\_empty(s)}
\end{defn}

\begin{note}
When popping from a stack, only the value that the stack holds should be returned, and the node that was on the stack should be deallocated.
\begin{lstlisting}[language=C]
int pop(Stack s) {
  assert(!is_empty(s));
  int ret - s->topnode->item;
  struct llnode * backup = s->topnode;
  s->topnode = s->topnode->next;
  free(backup);
  return ret;
}
\end{lstlisting}
\end{note}

\begin{defn}
A \textbf{queue} is a lineup where new items go to the abck of the line, and items are removed from the front of the line. Queue follows FIFO.\n
Operations: \textbf{add\_back(q,i), remove\_front(q), front(q), is\_empty(q)}
\end{defn}

\begin{defn}
The \textbf{sequence} ADT is useful when you need to retrieve, insert or delete an item at any position in the sequence. \n
Operations: \textbf{item\_at(s,k),insert\_at(s,k,i), remove\_at(s,k), length(s)}
\end{defn}

\begin{defn}
The \textbf{dictionary} ADT is a collection of \textbf{pairs} of \textbf{keys} and \textbf{values}. Each key is unique and has a corresponding value, but values may not be unique. Dictionaries are unsequenced, and useful when quick lookups are required.\n
Operations: \textbf{lookup(d,k), insert(d,k,v), remove(d,k)}
\end{defn}

\begin{defn}
The \textbf{set} ADT is similar to a mathematical set. It is a dictionary with no values, unsequenced, and usually sorted.\n
Operations: \textbf{member(s,i), add(s,i), union(s1,s2), intersection(s1,s2), difference(s1,s2), is\_subset(s1,s2), size(s)}
\end{defn}

\begin{note}
Typical implementations for the ADTs
\begin{itemize}
\item \textbf{Stack}: Linked list/ dynamic array
\item \textbf{Queue}: Linked list
\item \textbf{Sequence}: Linked list/ dynamic array
\item \textbf{Dictionary/Set}: Self-balancing BST/ Hash table
\end{itemize}
\end{note}

\begin{exmp}
Sometimes it is necessary for a stack to support types other than just integers. One way is to use typedef.
\begin{lstlisting}[language=C]
//item.h
typedef int ItemType // can be anything
\end{lstlisting}
Then the ADT module would implement stuff like
\begin{lstlisting}[language=C]
struct llnode {
  ItemType item;
  struct llnode * next;
}
void push(Stack s, ItemType i)  {...}
\end{lstlisting}
However, this method still only allows one type to be used with the stack.
\end{exmp}

\begin{defn}
The \textbf{void pointer} can point to any type, and are esentially just memory addresses. They can be converted, but they cannot be directly dereferenced. The void pointers must be converted to a different pointer before dereferencing.
\begin{lstlisting}[language=C]
int i = 42;
void *vp = &i;
int j = &vp; // Invalid

int *ip = vp;
int k = *ip; // Valid
int l = *(int *) vp; // Also valid, known as casting
\end{lstlisting}
\end{defn}

\begin{note}
Dictionary and set ADTs often sort and compare items, which is a problem with void pointers. A solution is to use a \textbf{comparison function} (pointer) stored in the ADT so that it would just call the function whenever comparison is necessary.
\end{note}
\begin{lstlisting}[language=C]
typedef int (* DictKeyCompare) (const void * ,const void * );

// create a dictionary that uses key comparison function f
Dictionary create_Dictionary (DictKeyCompare f);
\end{lstlisting}
\begin{qte}
In the implementation, the function pointer must be stored in the ADT. The comparison function would return -1 if $a < b$, 0 if $a == b$, and 1 if $a > b$.
\end{qte}

\begin{exmp}
The lookup function would then utilize the comparison function to compare.
\begin{lstlisting}[language=C]
void * lookup (Dictionary d, void * k) {
  struct bstnode *t = d->root;
  while (t) {
    int result = d->key_compare(k,t->key);
    if (result < 0) {
      t = t->left;
    } else if (result < 0) {
      t = t->right
    } else {
      return t->value; // Found
    }
  }
  return NULL; // Not in dictionary
}
\end{lstlisting}
\end{exmp}

\todo{Above and Beyond section, not tested but looks pretty useful}

\end{document}
