\documentclass[english, 12pt]{article}
\usepackage{yingconfig}

% ========================Variables======================================
\newcommand{\coursecode}{AFM 121}
\newcommand{\coursename}{Global Finacial Markets}
\newcommand{\thisprof}{Professor G. Blair}
\newcommand{\curterm}{Winter 2014}

\begin{document}
\notesheader
\section{Overview of Financial Systems}
Money can't buy happiness but it sure can rent it for a long time.
\begin{defn}
\textbf{Finance} is a branch of economics concerned wth resource allocation, management acquisition, and investment. Art of managing assets.
\end{defn}
\begin{qte}
There's nothing better to do than making money by doing absolutely jack.
\end{qte}
Banks attract deposits (safe, interest), and issue loans (diversity risk, interest).
\begin{defn}
\textbf{Spread} is the difference between deposit interest and loan interest for banks.
\end{defn}

\begin{note}
Investment banking offer transaction and advisory services for a commission.
\end{note}


\begin{defn}
A \textbf{market} is for lenders and investors to provide capital to borrowers and businesses.
\end{defn}
\begin{defn}
\textbf{Organic cash flow} is what a company generates on its own.
\end{defn}

\begin{qte}
Financial markets allow buyers and sellers to come together.
\end{qte}
\begin{note}
Financial facilitate price formation, and increase competition amongst buyers and sellers. Too much power can lead to decreased competition.

\begin{itemize}
\item Physical mechanism for money and ownership to transfer from buyer to seller.
\item Aims to increase economic efficiency.
\begin{itemize}
\item Reduce search costs
\item Reduce transaction costs
\item Diversification of risk
\end{itemize}
\end{itemize}
\end{note}

\begin{defn}
\textbf{Financial intermediaries} provide services to buyers and sellers. They also provide liquidity by buying or selling as needed to bridge the time difference between buyer and seller arrival.
\begin{itemize}
\item Banks, Trust companies, Investment Banks, Asset Management Firms
\item Individuals may not have the time to perform adequate due diligence. Financial intermediaries analyze the investments for a fee.
\end{itemize}
\end{defn}

\begin{defn}
\textbf{Moral hazard}: Buyers and sellers may have different incentives and financial intermediaries.
\end{defn}

\begin{defn}
\textbf{Financial regulation}: OSFI oversees Canadian banks. Focus on proficiency, assets, and activities.
\end{defn}

\begin{defn}
\textbf{Central banks} are responsible for the entire financial system. Controls money supply through open market operations and overnight interest rates. Lender of last resort.
\end{defn}
\begin{qte}
Central banks have direct impact on unemployment and the economy.
\end{qte}

\section{The Corporation}
\begin{defn}
\textbf{Sole proprietorship} is a business owned and run by one person.\\
Advantages: Easy to create\\\\
Disadvantages:
\begin{itemize}
\item Unlimited liability
\item Same entity
\item Limited life
\item Difficult to transfer ownership
\end{itemize}
\end{defn}
\begin{defn}
\textbf{Partnership} is similar to a sole proprietorship but with more than one owner.
\begin{itemize}
\item Income is taxed at personal level
\item All partners have unlimited personal liability
\item Partnership ends with death or withdrawal of any single partner.
\end{itemize}
\end{defn}
\begin{qte}
A partnership consists of both general partners and limited partners.
\end{qte}
\begin{defn}
\textbf{General partners} have the same rights and liability as partners in a regular partnership.
\end{defn}
\begin{defn}
\textbf{Limited partners} have limited liability.
\begin{itemize}
\item Death or withdrawal does not dissolve partnership
\item Interest in business is transferable
\item Have no management authority and cannot be legally involved in managerial decisions.
\end{itemize}
\end{defn}
\begin{defn}
A \textbf{corporation} is a legal entity separate from its owners. The corporation is solely responsible for its own obligations, and owners are not liable. Corporations must be legally formed.
\begin{itemize}
\item Ownership of a corporation is represented by shares of stock
\item Sum of all ownersip value is called equity
\item No limit to number of shareholders
\item Owners are able to receive dividends.
\item Dividends are taxed twice (Corporate and personal tax)
\end{itemize}
\end{defn}
\todo[inline]{Create a table instead}
\begin{defn}
\textbf{Income trusts} are flow through entities where all income produced by the business flows to the investors, and no earnings are retained in the business. They are not taxed. In 2006, government changed taxation of businesses and income trusts are taxable. Real Estate Income Trusts continue to have no tax at the business level.
\end{defn}
The board of directors directly control the corporation, not owners.\\\\
Shareholder wealth maximization is the goal that generally unites shareholders.
\subsubsection*{Principle-Agent Problem}
\begin{itemize}
\item Separation of ownership and control
\item Managers may act in their own interest
\item Solution is to tie management's compensation to firm performance.
\end{itemize}
\begin{defn}
\textbf{Hostile takeover}: Low stock prices may entice a corporate raider to buy enough stock so that they have enough contol to replace the current manaegment.
\end{defn}
\begin{defn}
\textbf{Stakeholders} are those who have an interest in the corporation, and includes shareholders and debt holders.
\begin{itemize}
\item Employees
\item Customers
\item Suppliers
\item Community
\end{itemize}
\end{defn}
\begin{defn}
\textbf{Corporate Bankruptcy Process:}
\begin{itemize}
\item Reorganization
\item Liquidation
\item Debt holders
\item Equity holders
\end{itemize}
\end{defn}
\begin{defn}
\textbf{Primary market} is the initial transaction between corporation and investors.
\end{defn}
\begin{defn}
\textbf{Secondary market} is the trades of existing stock between investors.
\end{defn}

\section{Raising Capital}
\begin{defn}
\textbf{Angel investors} are individual investors who buy equity in small private firms.
\end{defn}
\begin{defn}
\textbf{Venture capital firm} is a limited partnership that specializes in raising money to invest in the private equity of young firms. usually charge a substantial fee (usually 20\%).
\end{defn}
\begin{defn}
\textbf{Venture capitalist} is a general partner in the venture capital firm.
\end{defn}
\begin{defn}
\textbf{Private equity firms} are similar to venture capital firm but invest in more established firms.
\end{defn}
\begin{defn}
\textbf{Institutional investors} (eg: Pension funds) are active investors in private companies.
\end{defn}
\begin{defn}
\textbf{Sovereign Wealth Funds} are pools of money controlled by a government, and play an active role in the private equity market. Largest limited partners in global private equity markets. Usually raised from royalty, or taxes.
\end{defn}

\begin{defn}
An \textbf{underwriter} is an investment banking firm that manages a security issuance and designs its structure. Eg: IPO.
\end{defn}

\begin{defn}
\textbf{Exit strategy} is how investors will realize the return from their investmnts.
\begin{itemize}
\item Acquisition
\item IPO
\end{itemize}
\end{defn}

\begin{defn}
An IPO is the first time a company sells shares.
\begin{itemize}
\item \textbf{Primary offering:} new shares
\item \textbf{Existing shares:} existing shares
\end{itemize}
\end{defn}

\subsubsection*{Types of Offerings}
\begin{defn}
Best-Efforts: Underwriter does not guarantee stock is sold, but tries to sell the stock for the best price. Typically all or nothing.
\end{defn}
\begin{defn}
Fire Commitment: Agreement between underwriter and an issuing firm in which the underwriter guarantees that all shares are sold.
\end{defn}
\begin{defn}
Auction IPO: Takes bids from investors and then sets price that clears the market.
\end{defn}

\section{Buying and Selling Securities}
\subsubsection*{Types of Brokers}
\begin{itemize}
\item Full-service brokers
\item Discount brokers
\item Deep-discount brokers
\item Online brokers - Provide investment information, and allow customers to place buy and sell orders over the internet.
\end{itemize}
\begin{qte}
When dealing with brokers, advice is not expected. Legal duty to act in customer's best interest. Any disputes will be settled by arbitration.
\end{qte}
\begin{defn}
\textbf{Canadian Investor Protection Fund}: Is an insurance fund covering investors' brokerage accounts when member firms experience financial difficulties
\end{defn}
\subsubsection*{Types of Brokerage Accounts}
\begin{itemize}
\item \textbf{Cash account} - Securities are paid for in full
\item \textbf{Margin account} - securities can be bought and sold short on credit
\end{itemize}

\begin{defn}
The \textbf{margin} is the portion of the value of an investment that is not borrowed.
\end{defn}

\begin{defn}
The interest on the borrowed portion is the broker's \textbf{call money rate}.
\end{defn}

\begin{defn}
The \textbf{maintenance margin} is the margin amount that must be present at all times. When the margin drops below it, the broker can demand more funds through a \textbf{margin call}.
\[ \text{Margin} = \f{\text{Mkt value of shares} - \text{Margin loan}}{\text{Market value of shares}}\]

\[\text{Maintenance margin level} = \f{P \cdot \text{\# of shares} - \text{Amount borrowed}}{P \cdot \text{ \# of shares}}\]
\[P = \f{\text{borrowed} / \text{\# of shares}}{1-\text{Maintenance margin level}}\]
\end{defn}

\subsection{Margin on Purchase}
\begin{exmp}
Your margin account requires:
\begin{itemize}
  \item an initial margin of 50\%, and
  \item a maintenance margin of 30\%
  \item A Share in Miller Moore Equine Enterprises (MM) is selling for \$50.
  \item You have \$20,000, and you want to buy as much MM as you can.
  \item You may buy up to \$20,000 / 0.5 = \$40,000 worth of MM
\end{itemize}
\begin{center}
\begin{tabular}{|ll|ll|}
\hline
Assets & & Liabilities \& OE & \\
\hline
800 Shares & 40000 & Margin Loan & 20000 \\
& & Account Equity & 20000 \\
\hline
Total & 40000 & Total & 40000\\
\hline
\end{tabular}
\end{center}
If shares fall to \$35, then the new market value of the shares is $800 \times 35 = 28000$.
\begin{center}
\begin{tabular}{|ll|ll|}
\hline
Assets & & Liabilities \& OE & \\
\hline
800 Shares & {\color{red}28000} & Margin Loan & 20000 \\
& & Account Equity & {\color{red} 8000} \\
\hline
Total & {\color{red} 28000} & Total & {\color{red} 28000}\\
\hline
\end{tabular}
\end{center}
New margin would become $\f{8000}{28000} \approx 29\%$. Since this value is lower than the maintenance margin, a margin call would be expected.
\begin{note}
The required margin would be $30\% \times 28000 = 8400$. The investor would be required to pay $\$400$ in order to keep the security. As a result, the margin loan would drop to $\$19600$.
\end{note}
\end{exmp}
\begin{qte}
When calculating the percentage that was gained or lost, the numbers are based off of the invested amount, not the total (investment + loan). In the previous example, any gain or loss would be calculated based on the $20000$ invested, and not the $40000$.
\end{qte}

\begin{exmp}
Suppose you want to buy 300 shares of Pepsico, Inc. (PEP) at \$55 per share.
\begin{itemize}
\item Total cost: \$16,500
\item You have only \$9,900 so you must borrow \$6,600.
\item Your initial margin is \$9,900/\$16,500 = 60\%.
\item Suppose your maintenance margin is 40\%. At what price will you receive a margin call?
\end{itemize}
\begin{sol}
Here, the asset amount is unknown, and must be calculated. Since maintenance margin is $40\%$, then loan represents $60\%$. Assets would be $\f{6600}{0.6} = 11000$. Now drawing the table (technically the table isn't needed, but it's easier to see this way):
\begin{center}
\begin{tabular}{|ll|ll|}
\hline
Assets & & Liabilities \& OE & \\
\hline
300 Shares &{\color{red} 11000} & Margin Loan & 6600 \\
& & Account Equity & {\color{red} 4400} \\
\hline
Total & {\color{red} 11000} & Total & {\color{red} 11000}\\
\hline
\end{tabular}
\end{center}
Solving for an individual share: $\f{11000}{300} = \$36.67$.
\end{sol}
\end{exmp}
\begin{defn}
\textbf{Hypothecation} is the act of pledging securities as a collateral against a loan.
\end{defn}
\begin{defn}
\textbf{Street name registration} is an arrangement where the broker registers as the owner of the security.
\end{defn}
\begin{defn}
An investor takes a \textbf{long} position if he/she expects the price of a stock to go up. A \textbf{short} position anticipates the price to go down.
\end{defn}
\begin{defn}
\textbf{Short interest} is the amount of common stock held in short positions.
\end{defn}

\subsection{Short Selling}
\begin{qte}
Always draw out a balance sheet for easier calculations.
\end{qte}


\begin{exmp}
An example for short selling:
\begin{itemize}
\item You short 100 shares of Verizon Communications (VZ) at \$30 per share.
\item Your broker has a 50\% initial margin and a 40\% maintenance margin on short sales.
\item The value of stock borrowed that will be sold short is: \$30 × \$100 = \$3,000
\end{itemize}

\begin{center}
\begin{tabular}{|ll|ll|}
\hline
Assets & & Liabilities \& OE & \\
\hline
Sales Proceeds & 3000 & Short Position & 3000 \\
Initial Margin Deposit & {\color{red} 1500} & Account Equity & {\color{red} 1500} \\
\hline
Total & 4500 & Total & 4500\\
\hline
\end{tabular}
\end{center}
If price goes down to $\$20$, then the new short position would be at $100 \times 20 = 2000$.
\begin{note}
Assets and liabilities will remain the same.
\end{note}
\begin{center}
\begin{tabular}{|ll|ll|}
\hline
Assets & & Liabilities \& OE & \\
\hline
Sales Proceeds & 3000 & Short Position & {\color{red} 2000} \\
Initial Margin Deposit & 1500 & Account Equity & {\color{red} 2500} \\
\hline
Total & 4500 & Total & 4500\\
\hline
\end{tabular} \\
Margin: $\f{2500}{2000} = 125\%$
\end{center}
\end{exmp}
\begin{exmp}
If the price goes up to $\$40$, then the short position would be at $100 \times 40 = 4000$.
\begin{center}
\begin{tabular}{|ll|ll|}
\hline
Assets & & Liabilities \& OE & \\
\hline
Sales Proceeds & 3000 & Short Position & {\color{red} 4000} \\
Initial Margin Deposit & 1500 & Account Equity & {\color{red} 500} \\
\hline
Total & 4500 & Total & 4500\\
\hline
\end{tabular}
\end{center}
The margin would be at $\f{500}{4000} = 12.5\%$. The required margin would be $40\% \times 4000 = 1600$. The investor would have to pay $\$900$ to keep the short sale. This amount would be added to initial margin deposit.
\end{exmp}

\subsubsection*{Stock Market Order Types}
\begin{itemize}
\item Market order
\item Limit order
\item Stop order - Convert to market order once certain price is reached
\item Stop limit order - Convert to limit order once certain price is reached
\end{itemize}

\section{Market Efficiency}
According to CAPM, investors should hold risk-free assets in combination with the market portfolio of all risky securities.
\subsubsection*{Behaviour of Individual Investors}
\begin{itemize}
\item Usually have an underdiversification and portfolio bias. Familiarity Bias. Relative Wealth Concerns.
\item Overconfidence Bias. Sensation seeking.
\item If individuals depart from the CAPM in random ways, then the departures will most likely cancel out. Individuals will also hold market portfolio (no diversification) in aggregate, and there will be no effect on market prices.
\item Individuals are more likely to buy companies that are in the news, or had extreme returns. Stock returns tend to be higher on a sunny day at the location of the stock exchange.
\item Takeover Offers: Usually companies have to pay premium to takeover company. Price will likely go up.
\end{itemize}
\begin{defn}
\textbf{Disposition Effect:} Investors holding on to losing stocks and selling stocks that made a gain.
\end{defn}

\section{Time Value of Money}

\begin{qte}
Solve for the unknown between present value, future value, \# periods, periodic interest, and periodic payment.
\end{qte}
\begin{defn}
\textbf{Time value of money} is the equivalent value of two cash flows and two different points in time.
\end{defn}
\begin{qte}
It is only possible to compare or combine values at the same point in time.
\end{qte}
\begin{thrm}
Future value and present value:
\[FV = C \cdot (1+r)^n, PV = \f{C}{(1+r)^n}\]
\end{thrm}

\begin{defn}
\textbf{Net Present Value} is the cash inflows (benefits) subtract the cash outflows (costs).
\end{defn}

\begin{defn}
The value of a \textbf{perpetuity} is the cash flow divided by the interest rate. $PV = \f{C}{r}$.
\end{defn}

\begin{thrm}
\textbf{Present value of an annuity formula}:
\[PV=P\left(1-\f{1}{(1+r)^n}\right)\]
If there is no payment upfront,
\[PV = C \times \f{1}{r} \left(1-\f{1}{(1+r)^n}\right)\]
\end{thrm}
\begin{thrm}
Future value of an annuity:
\[C \times \f{1}{r} \left((1+r)^n - 1\right)\]
\end{thrm}

\begin{defn}
If an investment is \textbf{growing}, then there is a market rate of growth attached as well. This growing interest must be subtracted when calculating present value.
\end{defn}

\begin{thrm}
Present value of a growing perpetuity:
\[PV = \f{C}{r-g}\]
Present value of a growing annuity:
\[PV = \f{C}{r-g} \left(1 - \left(\f{1+g}{1+r}\right)^n\right)\]
\end{thrm}
\begin{defn}
Future value at a time of last payment of an n-period growing annuity:
\[ FV = \f{C}{r-g} \left\lbrack (1+r)^n - (1+g)^n\right\rbrack\]
We can just memorize PV formula, and compound it to obtain FV.
\end{defn}

\begin{defn}
The \textbf{internal rate of return} is the interest rate that sets the net present value of cash flows equal to zero.
\end{defn}

\section{Interest Rates}

\begin{defn}
The \textbf{Effective Annual Rate} is equivalent to the percentage that will calculate the total amount of interest that will be earned if interest is compounded annually.
\end{defn}

\begin{note}
To convert EAR to a monthly rate, $R = (1 + EAR)^{\f{1}{12}} - 1$.
\end{note}

\begin{defn}
\textbf{The annual percentage rate} indicates the amount of simple interest earned in one year.
\end{defn}

\begin{note}
To convert APR to monthly rate, $R = \f{APR}{12}$
\end{note}

\begin{note}
To determine how much of a payment was dedicated towards principal and how much towards interest, subtract the outstanding balance from principal. This number will be money dedicated towards principal. Then subtract this number from the sum of the payments so far, and it'll be the amount dedicated towards interest payment.
\end{note}

\section{Bonds}

\begin{defn}
Basic terminology:
\begin{itemize}
\item A \textbf{bond certificate} states the terms of the bond.
\item The \textbf{maturity date} is the final repayment date.
\item The \textbf{term} is the time remaining until repayment date.
\item The \textbf{coupon} is the promised interest payments. This is given as APR.
\item \textbf{Face value} is the notional amount used to compute interest payments. In theory, it is what the bond should be worth at the end of the interest payments.
\item The \textbf{coupon payment} will be the same every period. It is simply the face value multiplied by the coupon rate.
\end{itemize}
\end{defn}

\begin{defn}
\textbf{Zero coupon bonds} do not make coupon payments, so they are discounted more than regular bonds. \textbf{Treasury Bills} are zero-coupon bonds with a maturity of up to a year.
\end{defn}

\begin{defn}
The \textbf{yield to maturity} is the discount rate that sets the PV of the bond payments equal to the current market price of a bond.
\end{defn}

\begin{defn}
\textbf{Spot Interest Rate} is another term for a defalt-free, zero-coupon yield.
\end{defn}

\begin{defn}
\textbf{Coupon bonds} pay regular coupon interest payments, and pay the face value at maturity.
\end{defn}

\begin{note}
To solve for coupon bonds, RATE is YTM, NPER is the \# of periods, PMT is coupon, PV is negative of value of bond currently, and FV is the face value.
\end{note}

\begin{qte}
Make sure both coupon payments and yield to maturity are the same time frame!
\end{qte}

\begin{defn}
\textbf{Bond price:}
\begin{itemize}
\item \textbf{Discount} - Price is less than FV
\item \textbf{Par} - Price is equal to FV.
\item \textbf{Premium} - Price is greater than FV
\end{itemize}
\end{defn}

\begin{note}
There is an inverse relationship between interest rates and bond prices.
\end{note}

\begin{qte}
The sensitivity of a bond's price to changes in interest rates is directly correlated to the bond's \textbf{duration}.
\end{qte}

\section{Financial Statements}

\begin{defn}
\textbf{Financial statements} are firm-issued accounting reports with past performance information. They are filed with the provincial securities commission. Interim financial statements are quarterly while annual reports are annual.
\end{defn}

\begin{defn}
\textbf{Auditor} is a neutral third party that checks the validity of a firm's financial statements.
\end{defn}

\subsubsection*{Types of Financial Statements}
\begin{itemize}
\item Balance Sheet
\item Statement of Comprehensive Income
\item Statement of Cash Flows
\item Statement of Changes in Equity
\item Notes including accounting policies
\end{itemize}

\subsection{Balance Sheet}
\begin{defn}
A \textbf{balance sheet} is a snapshot at a specific point in time of a firm's financial position.
\end{defn}
\[ \text{Assets}  = \text{Liabilities} + \text{Shareholder's Equity}\]
\begin{itemize}
\item \textbf{Assets:} Current assets (cash, securities, inventory, prepaids), long-term assets (PPE \& goodwill) \\
\item \textbf{Liabilities:} Current liabilities (current portion of long term-debt, payables), long-term liabilities (lease,bond)
\item \textbf{Shareholder's Equity:} Book value may be negative. Market value is the value of the shares outstanding and cannot be negative.
\end{itemize}


\begin{rto}
\[\text{Net Working Capital} = \text{Current assets} - \text{current liabilities}\]
\end{rto}

\begin{defn}
\textbf{Liquidation value} is the value of the firm if all assets were sold and liabilities were paid.
\end{defn}

\begin{rto}
\[ \text{Market to book ratio} = \f{\text{Market Value of Equity}}{\text{Book Value of Equity}}\]
\end{rto}

\begin{rto}
\[\text{Debt-Equity Ratio} = \f{\text{Total Debt}}{\text{Total Equity}}\]
\end{rto}

\begin{rto}
\[ \text{Enterprise Value} = \text{Market Value of Equity} + \text{Debt} - \text{Cash}\]
\end{rto}

\begin{rto}
\[\text{Current Ratio} = \f{\text{Current Assets}}{\text{Current Liabilities}}\]
\[\text{Quick Ratio} = \f{\text{Current Assets - Inventories}}{\text{Current Liabilities}}\]
\end{rto}

\subsection{Income Statement}
\begin{defn}
The \textbf{income statement} lists a firm's revenues and expenses over a period of time.
\end{defn}

\begin{rto}
\[ \text{EPS} = \f{\text{Net Income}}{\text{Shares Outstanding}}\]
\end{rto}

\subsubsection*{Profitability Ratios}
\[ \text{Gross Margin} = \f{\text{Gross profit}}{\text{Sales}}\]
\[ \text{Operating Margin} =  \f{\text{Operating Income}}{\text{Sales}}\]
\[ \text{Net Profit Margin} = \f{\text{Net Income}}{\text{Total Sales}}\]

\begin{rto}
\[\text{Asset Turnover} = \f{\text{Total Sales}}{\text{Total Assets}}\]
\end{rto}

\begin{rto}
\[\text{Accounts Receivable Days} = \f{\text{Accounts Receivable}}{\text{Average Daily Sales}}\]
\end{rto}

\begin{defn}
\textbf{EBITDA} (earnings before interest, taxes, depreciation, and amortization) reflects the cash a firm has earned from its operations.
\end{defn}

\subsubsection*{Leverage Ratios}
\begin{itemize}
\item Operating income / Interest expense
\item EBIT / Interest Expense
\item EBITDA / Interest Expense
\end{itemize}

\subsubsection*{Investment Returns}
\[\text{Return on Equity} = \f{\text{Net Income}}{\text{Book Value of Equity}}\]
\[\text{Return on Assets} = \f{\text{Net Income}}{\text{Total Assets}}\]

\begin{rto}
\[ \text{Price Earning Ratio} = \f{\text{Market Capitalization}}{\text{Net Income}} = \f{\text{Share Price}}{\text{Earnings per Share}}\]
\end{rto}

\subsection{Statement of Cash Flows}
\begin{itemize}
\item \textbf{Operating Activities:} Adjusts net income by non-cash items relating to operating activities.
\item \textbf{Investing Activities:} Capital expenditure \& marketable securities
\item \textbf{Financing Activities:} Changes in borrowing, payments of dividends
\end{itemize}

\subsection{Other Financial Statement Information}
\begin{itemize}
\item Management Discussion and Analysis (MD\&A)
\item Statement of Shareholder's Equity
\item Statement of Comprehensive Income
\item Notes to the Financial Statements
\end{itemize}

------------------ MIDTERM ENDS HERE -------------------------------------
Half mc, half problems
90 mins
All on computer
\section{Valuing Stocks}

\subsection{Dividend-Discount Model}
Potential cash flows for an investment include the sale of stock (capital gain), and the dividend.

\begin{exmp}
Suppose you expect a stock to pay dividends of $0.56$ per share and trade for $45.60$ per share at the end of the year. If the investments with equivalent risk has an expected return of $6.8$, what is the most you would pay?
\begin{sol}
\[P_{0} = \f{Div_{1} + P_{1}}{1 + r_{g}} = \f{0.56 + 45.60}{1.0680} = 43.13\]
\[\text{Dividend Yield} = \f{1.92}{43.13}\]
\[\text{Capital Gains Yield} = \f{45.6 - 43.13}{43.13}\]
\end{sol}
\end{exmp}

\begin{defn}
In a Constant Divdend Growth Model, the price is equivalent to
\[P_{0} = \f{Div_{1}}{r_{E} - g}\]
\end{defn}

\begin{defn}
The dividend discount model discounts both the dividend and the capital gain to predict the current price of a stock.
\[P_{0} = \f{Div_{1}}{1 + r_{g}} + \f{Div_{2}}{(1 + r_{g})^2} + \cdots + \f{Div_{n}}{(1 + r_{g})^n} + \f{P_{1}}{(1 + r_{g})^n}\]
\begin{note}
The dividend here is the dividend at the end of \textbf{year 1}, not year 0.
\end{note}
\end{defn}

\begin{qte}
Most banks in Canada have a dividend payout ratio of around 40-50 percent.
\end{qte}

\subsubsection*{A Simple Model of Growth}
Assuming shares outstanding is constant, the firm can increase dividends in three ways
\begin{itemize}
\item Increase net income
\item Increase dividend payout ratje
\item Decrease shares outstanding
\end{itemize}
\end{document}