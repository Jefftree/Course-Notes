\documentclass[english, 12pt]{article}
\usepackage{yingconfig}

% ========================Variables======================================
\newcommand{\coursecode}{AFM 121}
\newcommand{\coursename}{Global Finacial Markets}
\newcommand{\thisprof}{Professor G. Blair}
\newcommand{\curterm}{Winter 2014}

\begin{document}
\notesheader
\section{Introduction}
Money  can't buy happiness but it sure can rent it for a long time.



\begin{defn}
\textbf{Finance} is a branch of economics concerned wth resource allocation, management acquisition, and investment. Art of managing assets.
\end{defn}
\begin{qte}
There's nothing better to do than making money by doing absolutely jack
\end{qte}
Unlikely for cumulative final.

Banks attract deposits (safe, interest), and issue loans (diversity risk, interest).
\begin{defn}
\textbf{Spread} is the difference between deposit interest and loan interest for banks.
\end{defn}


Investment banking offer transaction and advisory services for a commission.

\begin{defn}
A \textbf{market} is for lenders and investors to provide capital to borrowers and businesses.
\end{defn}
\begin{defn}
\textbf{Organic cash flow} is what a company generates on its own.
\end{defn}

\begin{qte}
Financial markets allow buyers and sellers to come together.
\end{qte}
\begin{note}
Financial facilitate price formation, and increase competition amongst buyers and sellers. Too much power can lead to decreased competition.

\begin{itemize}
\item Physical mechanism for money and ownership to transfer from buyer to seller.
\item Aims to increase economic efficiency.
\begin{itemize}
\item Reduce search costs
\item Reduce transaction costs
\item Diversification of risk
\end{itemize}
\end{itemize}
\end{note}

\begin{defn}
\textbf{Financial intermediaries} provide services to buyers and sellers. They also provide liquidity by buying or selling as needed to bridge the time difference between buyer and seller arrival.
\begin{itemize}
\item Banks, Trust companies, Investment Banks, Asset Management Firms
\item Individuals may not have the time to perform adequate due diligence. Financial intermediaries analyze the investments for a fee.
\end{itemize}
\end{defn}

\begin{defn}
\textbf{Moral hazard}: Buyers and sellers may have different incentives and financial intermediaries.
\end{defn}

\begin{defn}
\textbf{Financial regulation}: OSFI oversees Canadian banks. Focus on proficiency, assets, and activities.
\end{defn}

\begin{defn}
\textbf{Central banks} are responsible for the entire financial system. Controls money supply through open market operations and overnight interest rates. Lender of last resort.
\end{defn}
\begin{qte}
Central banks have direct impact on unemployment and the economy.
\end{qte}


\begin{defn}
\textbf{Sole proprietorship} is a business owned and run by one person.\\
Advantages: Easy to create\\
Disadvantages
\begin{itemize}
\item Unlimited liability
\item Same entity
\item Limited life
\item Difficult to transfer ownership
\end{itemize}
\end{defn}
\begin{defn}
\textbf{Partnership} is similar to a sole proprietorship but with more than one owner.\\
\begin{itemize}
\item Income is taxed at personal level
\item All partners have unlimited personal liability
\item Partnership ends with death or withdrawal of any single partner.
\end{itemize}
\end{defn}
\begin{qte}
A partnership consists of both general partners and limited partners.
\end{qte}
\begin{defn}
\textbf{General partners} have the same rights and liability as partners in a regular partnership.
\end{defn}
\begin{defn}
\textbf{Limited partners} have limited liability.
\begin{itemize}
\item Death or withdrawal does not dissolve partnership
\item Interest in business is transferable
\item Have no management authority and cannot be legally involved in managerial decisions.
\end{itemize}
\end{defn}
\begin{defn}
A \textbf{corporation} is a legal entity separate from its owners. The corporation is solely responsible for its own obligations, and owners are not liable. Corporations must be legally formed. 
\begin{itemize}
\item Ownership of a corporation is represented by shares of stock
\item Sum of all ownersip value is called equity
\item No limit to number of shareholders
\item Owners are able to receive dividends.
\item Dividends are taxed twice (Corporate and personal tax)
\end{itemize}
\end{defn}
\todo[inline]{Create a table instead}
\todo[color=green!40]{And a green note}
\begin{defn}
\textbf{Income trusts} are flow through entities where all income produced by the business flows to the investors, and no earnings are retained in the business. They are not taxed. In 2006, government changed taxation of businesses and income trusts are taxable. Real Estate Income Trusts continue to have no tax at the business level.
\end{defn}
The board of directors directly control the corporation, not owners.\\\\
Shareholer wealth maximizationi s the goal that generally unites shareholders.
\subsubsection*{Principle-Agent Problem}
\begin{itemize}
\item Separation of ownership and control
\item Managers may act in their own interest
\item Solution is to tie management's compensation to firm performance.
\end{itemize}
\begin{defn}
\textbf{Hostile takeover}: Low stock prices may entice a corporate raider to buy enough stock so that they have enough contol to replace the current manaegment.
\end{defn}
\begin{defn}
\textbf{Stakeholders} are those who have an interest in the corporation, and includes shareholders and debt holders.
\begin{itemize}
\item Employees
\item Customers
\item Suppliers
\item Community
\end{itemize}
\end{defn}
\begin{defn}
\textbf{Corporate Bankruptcy Process:}
\begin{itemize}
\item Reorganization
\item Liquidation
\item Debt holders
\item Equity holders
\end{itemize}
\end{defn}
\begin{defn}
\textbf{Primary market} is the initial transaction between corporation and investors.
\end{defn}
\begin{defn}
\textbf{Secondary market} is the trades of existing stock between investors.
\end{defn}
--------------------------------------------------
\begin{defn}
\textbf{Angel investors} are individual investors who buy equity in small private firms.
\end{defn}
\begin{defn}
\textbf{Venture capital firm} is a limited partnership that specializes in raising money to invest in the private equity of young firms. sually charge a substantial fee (usually 20\%).
\end{defn}
\begin{defn}
\textbf{Venture capitalist} is a general partner in the venture capital firm. U
\end{defn}
\begin{defn}
\textbf{Private equity firms} are similar to venture capital firm but invest in more established firms.
\end{defn}
\begin{defn}
\textbf{Institutional investors} (eg: Pension funds) are active investors in private companies.
\end{defn}
\begin{defn}
\textbf{Sovereign Wealth Funds} are pools of money controlled by a government, and play an active role in teh private equity market. Largest limited partners in global private equity markets. Usually raised from royalty, or taxes.
\end{defn}



\begin{defn}
An \textbf{underwriter} is an investment banking firm that manages a security issuance and designs its structure. Eg: IPO.
\end{defn}


\begin{defn}
\textbf{Exit strategy} is how investors will realize the return from their investmnts.
\begin{itemize}
\item Acquisition
\item IPO
\end{itemize}
\end{defn}


\begin{defn}
An IPO is the first time a company sells shares.
\begin{itemize}
\item \textbf{Primary offering:} new shares
\item \textbf{Existing shares:} existing shares
\end{itemize}
\end{defn}



\subsubsection*{Types of OFferings}
\begin{defn}
Best-Efforts: Underwriter does not guarantee stock is sold, but tries to sell the stock for the best price. Typically all or nothing.
\end{defn}
\begin{defn}
Fire Commitment: Agreement between underwriter and an issuing firm in which the underwriter guarantees that all shares are sold.
\end{defn}
\begin{defn}
Auction IPO: Takes bids from investors and then sets price that clears the market.
\end{defn}
\todo{Slide 47 to end left}

\subsubsection*{Types of Brokers}
\begin{itemize}
\item Full-service brokers
\item Discount brokers
\item Deep-discount brokers
\item Online brokers - Provide investment information, and allow customers to place buy and sell orders over the internet.
\end{itemize}
\begin{defn}
Dealing with brokers: Advice is not expected. Legal duty to act in customer's best interest. Any disputes will be settled by arbitration.
\end{defn}
\begin{defn}
\textbf{Canadian Investor Protection Fund}: Is an insurance fund covering investors' brokerage accounts when member firms experience financial difficulties
\end{defn}
\subsubsection*{Types of Brokerage Accounts}
\begin{itemize}
\item Cash account - Securities are paid for in full
\item Margin account - securities can be bought and sold short on credit
\end{itemize}
\todo{Add Margin stuff, go over maintenance margin}
\subsubsection*{Financial Markets are Buying and Selling Securities III}
\begin{defn}
\textbf{Hypothecation} is the act of pledging securities as a collateral against a loan.
\end{defn}
\begin{defn}
\textbf{Street name registration} is an arrangement where the broker registers as the owner of the security.
\end{defn}
\begin{defn}
An investor takes a \textbf{long} position if he/she expects the price of a stock to go up. A \textbf{short} position anticipates the price to go down.
\end{defn}
\begin{defn}
\textbf{Short interest} is the amount of common stock held in short positions.
\end{defn}
\subsubsection*{Stock Market Order Types}
\begin{itemize}
\item Market order
\item Limit order 
\item Stop order - Convert to market order once certain price is reached
\item Stop limit order - Convert to limit order once certain price is reached
\end{itemize}

\subsubsection{Behaviour of Idividual Investors}
Usually have an underdiversification and portfolio bias. Familiarity Bias. Relative Wealth Concerns.

According to CAPM, investors should hold risk-free assets in combination with the market portfolio of all risky securities.
\subsubsection{Excessive trading \& overconfidence}
Overconfidence Bias. Sensation seeking.

If individuals depart from the CAPM in random ways, then the departures will most likely cancel out. Individuals will also hold market portfolio (no diversification) in aggregate, and there will be no effect on market prices.

Disposition Effect: Investors holding on to losing stocks and selling stocks that made a gain.

Individuals are more likely to buy companies th at are in the news, or had extreme returns.Stock returns tend to be higher on a sunny day at the location of the stock exchange.

Takeover Offers: Usually companies have to pay premium to takeover company. Price will likely go up.

\todo{Jim Cramer}
\end{document}


