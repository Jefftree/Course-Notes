\documentclass[english, 12pt]{article}
\usepackage{yingconfig}

% ========================Variables======================================
\newcommand{\coursecode}{STAT 230}
\newcommand{\coursename}{Probability}
\newcommand{\thisprof}{Professor N. Mohammad}
\newcommand{\curterm}{Winter 2014}

\begin{document}
\notesheader
\section{Introduction}
\begin{note}
Assignment are handed in on learn. Electronic copy only
\end{note}
\subsection{Introduction}
\begin{defn}
\textbf{Probability} is the study of randomnesses and uncertainty.
\begin{itemize}
\item Variability in population, processes or phenomena
\end{itemize}
\end{defn}
\begin{defn}
\textbf{Classic Interpretatation} makes assumptions about the physical world to deduce probability.
\[ \f{\text{\# ways an event can occur}}{\text{Total \# of outcomes}}\]
\end{defn}
\begin{defn}
\textbf{Relative-frequency interpretation} is when the probability of specific outcome is defined as the proportion of times it occurs over the long run. May only be used if the experiment may be repeated.
\end{defn}
\begin{defn}
\textbf{Frequency} is just the amount of times an event occurs while \textbf{relative frequency} is a fraction of the amount of times an event occurs over the total amount of possible outcomes.
\end{defn}
\begin{defn}
\textbf{Personal-probability interpretation} is the degree to which a given individual believes the event wil happen.
\end{defn}
\begin{qte}
\textbf{Coherent} means that personal probability of one event does not contradict personal probability of another.
\end{qte}
\begin{exmp}
If the probability of finding a parking space is 0.2, the probability of not finding one should be 0.8.
\end{exmp}
\subsection{Probability Models}
\begin{defn}
\textbf{Experiment} is any action, phenomenon, or process whose outcome is subject to uncertainty.
\end{defn}
\begin{defn}
\textbf{Trial} is a single repetition of an experiment.
\end{defn}
\begin{defn}
\textbf{Sample space}, denoted by $S$, is the set of possible distinct outcomes. In a single trial, only one outcome may occur. The sample space may be either \textbf{discrete} or \textbf{continuous}. 
\end{defn}
\begin{defn}
An \textbf{event} is any subset of outcomes contained in the sample space $S$. 
\begin{itemize}
\item \textbf{Simple} - one outcome
\item \textbf{Compound} - more than one outcome
\end{itemize}
\end{defn}
\subsection*{Probability notation}
$P(\text{event})$ is used to denote the probability of an event occurring. The \textbf{compliment} is the probability of an event not happening and is denoted as $P(A^c)$ or $P(\overline{A})$
\begin{defn}
The odds in \textbf{favour} of an event is the odds of an event occurring compared to its compliment.
\[\f{P(A)}{P(A^c)} = \f{P(A)}{1-P(A)}\]
The odds against is the reciprocal of odds in favour.
\end{defn}
\begin{thrm}
\[\sum_{i=0}^k P(A_{i}) = 1\]
The set $P(A_{i}), i = 1,2,\dots$ is the probability distribution on $S$.
\end{thrm}
\begin{thrm}
If $A$ and $B$ are two events wirh $A \subseteq B$, then $P(A) \leq P(B)$
\end{thrm}
\begin{defn}
Two events are \textbf{mutually exclusive} or \textbf{disjoint} if they cannot happen simultaneously.
\[A \cap B = \emptyset \]
\end{defn}
\tabularnewline
\subsection{Counting Techniques}
Counting Principle $\bullet$ Permutations $\bullet$ Combinations\\\\

\begin{defn}
\textbf{Addition rule}: When there are $m$ ways to perform $A$, and $n$ ways to perform $B$, there are $m+n$ ways to perform $A$ \textbf{OR} $B$.
\end{defn}
\begin{defn}
\textbf{Product rule}: Where there are $p$ ways to perform $A$, and $q$ ways to perform $B$, there are $m \times n$ ways to perform $A$ \textbf{AND} $B$.
\end{defn}
\begin{defn}
\textbf{Uniform Probability Model}:
\[P(A) = \f{\text{Outcomes in } A}{\text{Outcomes in } S} \]
\end{defn}

\begin{defn}
\textbf{Permutation} is an arrangement of elements in an ordered list.
\begin{note}
The amount of ways to arrange $n$ items (all of them have to be used) is $n!$
\end{note}
\[P_{k,n} = \f{n!}{(n-k)!}\]
\end{defn}

\begin{defn}
Any unordered sequence of $k$ objects taken from a set of $n$ distinct objects is called a \textbf{combination}.
\[C_{k,n} = {n \choose k} = \f{n!}{k! (n-k)!}\]
\end{defn}
\begin{exmp}
\textbf{Binomial Theorem}:
\[(1+x)^n = {n \choose 0} + {n \choose 1} x + {n \choose 2} x^2 + \dots + {n \choose n} x^n\]
\end{exmp}
\end{document}

